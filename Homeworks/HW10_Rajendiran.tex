\PassOptionsToPackage{unicode=true}{hyperref} % options for packages loaded elsewhere
\PassOptionsToPackage{hyphens}{url}
%
\documentclass[]{article}
\usepackage{lmodern}
\usepackage{amssymb,amsmath}
\usepackage{ifxetex,ifluatex}
\usepackage{fixltx2e} % provides \textsubscript
\ifnum 0\ifxetex 1\fi\ifluatex 1\fi=0 % if pdftex
  \usepackage[T1]{fontenc}
  \usepackage[utf8]{inputenc}
  \usepackage{textcomp} % provides euro and other symbols
\else % if luatex or xelatex
  \usepackage{unicode-math}
  \defaultfontfeatures{Ligatures=TeX,Scale=MatchLowercase}
\fi
% use upquote if available, for straight quotes in verbatim environments
\IfFileExists{upquote.sty}{\usepackage{upquote}}{}
% use microtype if available
\IfFileExists{microtype.sty}{%
\usepackage[]{microtype}
\UseMicrotypeSet[protrusion]{basicmath} % disable protrusion for tt fonts
}{}
\IfFileExists{parskip.sty}{%
\usepackage{parskip}
}{% else
\setlength{\parindent}{0pt}
\setlength{\parskip}{6pt plus 2pt minus 1pt}
}
\usepackage{hyperref}
\hypersetup{
            pdftitle={IST772-- Problem Set 10},
            pdfauthor={Sathish Kumar Rajendiran},
            pdfborder={0 0 0},
            breaklinks=true}
\urlstyle{same}  % don't use monospace font for urls
\usepackage[margin=1in]{geometry}
\usepackage{color}
\usepackage{fancyvrb}
\newcommand{\VerbBar}{|}
\newcommand{\VERB}{\Verb[commandchars=\\\{\}]}
\DefineVerbatimEnvironment{Highlighting}{Verbatim}{commandchars=\\\{\}}
% Add ',fontsize=\small' for more characters per line
\usepackage{framed}
\definecolor{shadecolor}{RGB}{248,248,248}
\newenvironment{Shaded}{\begin{snugshade}}{\end{snugshade}}
\newcommand{\AlertTok}[1]{\textcolor[rgb]{0.94,0.16,0.16}{#1}}
\newcommand{\AnnotationTok}[1]{\textcolor[rgb]{0.56,0.35,0.01}{\textbf{\textit{#1}}}}
\newcommand{\AttributeTok}[1]{\textcolor[rgb]{0.77,0.63,0.00}{#1}}
\newcommand{\BaseNTok}[1]{\textcolor[rgb]{0.00,0.00,0.81}{#1}}
\newcommand{\BuiltInTok}[1]{#1}
\newcommand{\CharTok}[1]{\textcolor[rgb]{0.31,0.60,0.02}{#1}}
\newcommand{\CommentTok}[1]{\textcolor[rgb]{0.56,0.35,0.01}{\textit{#1}}}
\newcommand{\CommentVarTok}[1]{\textcolor[rgb]{0.56,0.35,0.01}{\textbf{\textit{#1}}}}
\newcommand{\ConstantTok}[1]{\textcolor[rgb]{0.00,0.00,0.00}{#1}}
\newcommand{\ControlFlowTok}[1]{\textcolor[rgb]{0.13,0.29,0.53}{\textbf{#1}}}
\newcommand{\DataTypeTok}[1]{\textcolor[rgb]{0.13,0.29,0.53}{#1}}
\newcommand{\DecValTok}[1]{\textcolor[rgb]{0.00,0.00,0.81}{#1}}
\newcommand{\DocumentationTok}[1]{\textcolor[rgb]{0.56,0.35,0.01}{\textbf{\textit{#1}}}}
\newcommand{\ErrorTok}[1]{\textcolor[rgb]{0.64,0.00,0.00}{\textbf{#1}}}
\newcommand{\ExtensionTok}[1]{#1}
\newcommand{\FloatTok}[1]{\textcolor[rgb]{0.00,0.00,0.81}{#1}}
\newcommand{\FunctionTok}[1]{\textcolor[rgb]{0.00,0.00,0.00}{#1}}
\newcommand{\ImportTok}[1]{#1}
\newcommand{\InformationTok}[1]{\textcolor[rgb]{0.56,0.35,0.01}{\textbf{\textit{#1}}}}
\newcommand{\KeywordTok}[1]{\textcolor[rgb]{0.13,0.29,0.53}{\textbf{#1}}}
\newcommand{\NormalTok}[1]{#1}
\newcommand{\OperatorTok}[1]{\textcolor[rgb]{0.81,0.36,0.00}{\textbf{#1}}}
\newcommand{\OtherTok}[1]{\textcolor[rgb]{0.56,0.35,0.01}{#1}}
\newcommand{\PreprocessorTok}[1]{\textcolor[rgb]{0.56,0.35,0.01}{\textit{#1}}}
\newcommand{\RegionMarkerTok}[1]{#1}
\newcommand{\SpecialCharTok}[1]{\textcolor[rgb]{0.00,0.00,0.00}{#1}}
\newcommand{\SpecialStringTok}[1]{\textcolor[rgb]{0.31,0.60,0.02}{#1}}
\newcommand{\StringTok}[1]{\textcolor[rgb]{0.31,0.60,0.02}{#1}}
\newcommand{\VariableTok}[1]{\textcolor[rgb]{0.00,0.00,0.00}{#1}}
\newcommand{\VerbatimStringTok}[1]{\textcolor[rgb]{0.31,0.60,0.02}{#1}}
\newcommand{\WarningTok}[1]{\textcolor[rgb]{0.56,0.35,0.01}{\textbf{\textit{#1}}}}
\usepackage{graphicx,grffile}
\makeatletter
\def\maxwidth{\ifdim\Gin@nat@width>\linewidth\linewidth\else\Gin@nat@width\fi}
\def\maxheight{\ifdim\Gin@nat@height>\textheight\textheight\else\Gin@nat@height\fi}
\makeatother
% Scale images if necessary, so that they will not overflow the page
% margins by default, and it is still possible to overwrite the defaults
% using explicit options in \includegraphics[width, height, ...]{}
\setkeys{Gin}{width=\maxwidth,height=\maxheight,keepaspectratio}
\setlength{\emergencystretch}{3em}  % prevent overfull lines
\providecommand{\tightlist}{%
  \setlength{\itemsep}{0pt}\setlength{\parskip}{0pt}}
\setcounter{secnumdepth}{0}
% Redefines (sub)paragraphs to behave more like sections
\ifx\paragraph\undefined\else
\let\oldparagraph\paragraph
\renewcommand{\paragraph}[1]{\oldparagraph{#1}\mbox{}}
\fi
\ifx\subparagraph\undefined\else
\let\oldsubparagraph\subparagraph
\renewcommand{\subparagraph}[1]{\oldsubparagraph{#1}\mbox{}}
\fi

% set default figure placement to htbp
\makeatletter
\def\fps@figure{htbp}
\makeatother


\title{IST772-- Problem Set 10}
\author{Sathish Kumar Rajendiran}
\date{}

\begin{document}
\maketitle

Attribution statement: 1. I did this homework by myself, with help from
the book and the professor.

\begin{Shaded}
\begin{Highlighting}[]
\CommentTok{# import libraries }
\CommentTok{#create a function to ensure the libraries are imported}
\NormalTok{EnsurePackage <-}\StringTok{ }\ControlFlowTok{function}\NormalTok{(x)\{}
\NormalTok{  x <-}\StringTok{ }\KeywordTok{as.character}\NormalTok{(x)}
    \ControlFlowTok{if}\NormalTok{ (}\OperatorTok{!}\KeywordTok{require}\NormalTok{(x,}\DataTypeTok{character.only =} \OtherTok{TRUE}\NormalTok{))\{}
      \KeywordTok{install.packages}\NormalTok{(}\DataTypeTok{pkgs=}\NormalTok{x, }\DataTypeTok{repos =} \StringTok{"http://cran.us.r-project.org"}\NormalTok{)}
      \KeywordTok{require}\NormalTok{(x, }\DataTypeTok{character.only =} \OtherTok{TRUE}\NormalTok{)}
\NormalTok{    \}}
\NormalTok{  \}}
\end{Highlighting}
\end{Shaded}

\hypertarget{chapter-11-exercise-1}{%
\section{Chapter 11, Exercise 1}\label{chapter-11-exercise-1}}

\emph{Download and library the nlme package and use data(``Pixel'') to
activate the Pixel data set and ``? Pixel'' for documentation. Inspect
the data and create a box plot showing the Pixel intensities on
different days. (1 pt) Run a repeated measures ANOVA to compare pixel
intensities on days 4, 6, 10 and 14 using aov() (1 pt) and report the
results. (1 pt) You can use a command like:}

\hypertarget{question-2-repeated-measures-data}{%
\section{1) Question 2: Repeated Measures
data}\label{question-2-repeated-measures-data}}

\begin{itemize}
\tightlist
\item
  Data Exploration

  \begin{itemize}
  \tightlist
  \item
    glm(Catholic.b \textasciitilde{} Fertility +
    Agriculture,data=swiss,family = binomial(link = ``logit'')) is
    stored on a variable ``swiss.glm''
  \item
    above code runs a logistic regression analysis on Catholic.b as the
    dependent variable and Fertility and Agriculture as the predictors
    from ``swiss'' dataset.
  \end{itemize}
\item
  Balancing Data

  \begin{itemize}
  \tightlist
  \item
    From the glm(Catholic.b \textasciitilde{} Fertility +
    Agriculture,data=swiss,family = binomial(link = ``logit'')) ; this
    formula predicts ``Catholic.b'' values from Fertility and
    Agriculture combined from the dataset ``swiss''
  \item
    Procedure glm() is similar to lm() procedure and expects the
    dependent variable (variable, in question for prediction) first.
    Followed by, independent/prdictor variable and a link function.
  \end{itemize}
\item
  Repeated measures ANOVA

  \begin{itemize}
  \tightlist
  \item
    From the glm(Catholic.b \textasciitilde{} Fertility +
    Agriculture,data=swiss,family = binomial(link = ``logit'')) ; this
    formula predicts ``Catholic.b'' values from Fertility and
    Agriculture combined from the dataset ``swiss''
  \end{itemize}
\item
  Results

  \begin{itemize}
  \tightlist
  \item
    From the glm(Catholic.b \textasciitilde{} Fertility +
    Agriculture,data=swiss,family = binomial(link = ``logit'')) ; this
    formula predicts ``Catholic.b'' values from Fertility and
    Agriculture combined from the dataset ``swiss''
  \end{itemize}
\end{itemize}

\begin{Shaded}
\begin{Highlighting}[]
\KeywordTok{EnsurePackage}\NormalTok{(}\StringTok{"nlme"}\NormalTok{)}
\end{Highlighting}
\end{Shaded}

\begin{verbatim}
## Loading required package: nlme
\end{verbatim}

\begin{Shaded}
\begin{Highlighting}[]
\NormalTok{?Pixel}
\CommentTok{# X-ray pixel intensities over time}
\CommentTok{# Description}
\CommentTok{# The Pixel data frame has 102 rows and 4 columns of data on the pixel }
\CommentTok{# intensities of CT scans of dogs over time}

\CommentTok{# This data frame contains the following columns:}
\CommentTok{# Dog: a factor with levels 1 to 10 designating the dog on which the scan was made Side}
\CommentTok{# a factor with levels L and R designating the side of the dog being scanned}
\CommentTok{# day: a numeric vector giving the day post injection of the contrast on which the }
\CommentTok{# scan was made}
\CommentTok{# pixel}
\CommentTok{# a numeric vector of pixel intensities}

\KeywordTok{summary}\NormalTok{(Pixel)}
\end{Highlighting}
\end{Shaded}

\begin{verbatim}
##       Dog     Side        day            pixel     
##  1      :14   L:51   Min.   : 0.00   Min.   :1034  
##  2      :14   R:51   1st Qu.: 4.00   1st Qu.:1054  
##  3      :14          Median : 6.00   Median :1088  
##  4      :14          Mean   : 7.49   Mean   :1087  
##  5      :10          3rd Qu.:10.00   3rd Qu.:1110  
##  6      :10          Max.   :21.00   Max.   :1161  
##  (Other):26
\end{verbatim}

\begin{Shaded}
\begin{Highlighting}[]
\CommentTok{# dim(Pixel) # [1] 102   4}

\CommentTok{# View(Pixel)}
\CommentTok{# str(Pixel)}
\CommentTok{# colnames(Pixel)}

\CommentTok{# define variables}
\NormalTok{Dog <-}\StringTok{ }\NormalTok{Pixel}\OperatorTok{$}\NormalTok{Dog}
\NormalTok{Side <-}\StringTok{ }\NormalTok{Pixel}\OperatorTok{$}\NormalTok{Side}
\NormalTok{day <-}\StringTok{ }\NormalTok{Pixel}\OperatorTok{$}\NormalTok{day}
\NormalTok{pixel <-}\StringTok{ }\NormalTok{Pixel}\OperatorTok{$}\NormalTok{pixel}
\end{Highlighting}
\end{Shaded}

\begin{Shaded}
\begin{Highlighting}[]
\CommentTok{#box plot to compare the tensions}
\KeywordTok{boxplot}\NormalTok{(pixel}\OperatorTok{~}\NormalTok{day, }\DataTypeTok{data=}\NormalTok{Pixel,}
       \DataTypeTok{border=}\StringTok{"orange"}\NormalTok{, }
       \DataTypeTok{col=}\StringTok{"steelblue"}\NormalTok{,}
       \DataTypeTok{freq=}\OtherTok{FALSE}\NormalTok{,}
       \DataTypeTok{las=}\DecValTok{1}\NormalTok{, }
       \CommentTok{# breaks=10,}
       \CommentTok{# notch = TRUE,}
       \DataTypeTok{horizontal =} \OtherTok{FALSE}\NormalTok{ ,}\DataTypeTok{main=}\StringTok{" Distribution of pixel vs day"}\NormalTok{)}
\end{Highlighting}
\end{Shaded}

\includegraphics{HW10_Rajendiran_files/figure-latex/unnamed-chunk-3-1.pdf}

\begin{Shaded}
\begin{Highlighting}[]
\KeywordTok{EnsurePackage}\NormalTok{(}\StringTok{"tidyverse"}\NormalTok{)}
\end{Highlighting}
\end{Shaded}

\begin{verbatim}
## Loading required package: tidyverse
\end{verbatim}

\begin{verbatim}
## -- Attaching packages --------------------------------------------------------------------------------------------------------------------------------------------------------------------- tidyverse 1.3.0 --
\end{verbatim}

\begin{verbatim}
## v ggplot2 3.3.2     v purrr   0.3.4
## v tibble  3.0.1     v dplyr   1.0.0
## v tidyr   1.0.2     v stringr 1.4.0
## v readr   1.3.1     v forcats 0.5.0
\end{verbatim}

\begin{verbatim}
## -- Conflicts ------------------------------------------------------------------------------------------------------------------------------------------------------------------------ tidyverse_conflicts() --
## x dplyr::collapse() masks nlme::collapse()
## x dplyr::filter()   masks stats::filter()
## x dplyr::lag()      masks stats::lag()
\end{verbatim}

\begin{Shaded}
\begin{Highlighting}[]
\CommentTok{#data distribution}
\CommentTok{# violin plot with mean points}

\NormalTok{Pixel }\OperatorTok
\StringTok{  }\KeywordTok{pivot_longer}\NormalTok{(}\DataTypeTok{cols=}\OperatorTok{-}\KeywordTok{c}\NormalTok{(Dog,Side), }\DataTypeTok{names_to=}\StringTok{"variable"}\NormalTok{, }\DataTypeTok{values_to=}\StringTok{"value"}\NormalTok{, }\DataTypeTok{values_drop_na =} \OtherTok{TRUE}\NormalTok{) }\OperatorTok\StringTok{ }
\StringTok{  }\KeywordTok{ggplot}\NormalTok{(}\KeywordTok{aes}\NormalTok{(}\DataTypeTok{x=}\NormalTok{variable, }\DataTypeTok{y=}\NormalTok{value)) }\OperatorTok{+}\StringTok{ }\KeywordTok{geom_violin}\NormalTok{(}\DataTypeTok{trim=}\OtherTok{FALSE}\NormalTok{, }\DataTypeTok{fill=}\StringTok{'steelblue'}\NormalTok{, }\DataTypeTok{color=}\StringTok{"orange"}\NormalTok{) }\OperatorTok{+}\StringTok{ }
\StringTok{  }\KeywordTok{facet_wrap}\NormalTok{( }\OperatorTok{~}\StringTok{ }\NormalTok{variable, }\DataTypeTok{scales=}\StringTok{"free"}\NormalTok{) }\OperatorTok{+}\StringTok{ }
\StringTok{  }\KeywordTok{stat_summary}\NormalTok{(}\DataTypeTok{fun=}\NormalTok{mean, }\DataTypeTok{geom=}\StringTok{"point"}\NormalTok{, }\DataTypeTok{shape=}\DecValTok{23}\NormalTok{, }\DataTypeTok{size=}\DecValTok{4}\NormalTok{,}\DataTypeTok{color=}\StringTok{"white"}\NormalTok{)  }\OperatorTok{+}\StringTok{ }
\StringTok{  }\KeywordTok{theme_minimal}\NormalTok{()}
\end{Highlighting}
\end{Shaded}

\includegraphics{HW10_Rajendiran_files/figure-latex/unnamed-chunk-4-1.pdf}

\begin{Shaded}
\begin{Highlighting}[]
\CommentTok{# EnsurePackage("dlookr") # outlier analysis}
\CommentTok{# # #outlier analysis}
\CommentTok{# plot_outlier(Pixel)}
\end{Highlighting}
\end{Shaded}

\begin{Shaded}
\begin{Highlighting}[]
\CommentTok{# Corellation between Agriculture and Fertility}
\KeywordTok{plot}\NormalTok{(pixel, day}
\NormalTok{     ,}\DataTypeTok{main=}\StringTok{"Corellation between pixel and day"}
\NormalTok{     ,}\DataTypeTok{col=}\StringTok{"red"}\NormalTok{)}
\end{Highlighting}
\end{Shaded}

\includegraphics{HW10_Rajendiran_files/figure-latex/unnamed-chunk-6-1.pdf}

\emph{to subset the data. Keeping in mind that the data will need to be
balanced before you can conduct this analysis. (1 pt) Try running a
command like:}

\begin{Shaded}
\begin{Highlighting}[]
\CommentTok{#balanced Data}
\NormalTok{myPixel <-}\StringTok{ }\KeywordTok{subset}\NormalTok{(Pixel, day }\OperatorTok\StringTok{ }\KeywordTok{c}\NormalTok{(}\DecValTok{4}\NormalTok{,}\DecValTok{6}\NormalTok{,}\DecValTok{10}\NormalTok{,}\DecValTok{14}\NormalTok{))  }\CommentTok{# copy the dataset}
\NormalTok{myPixel}\OperatorTok{$}\NormalTok{dayFact <-}\StringTok{ }\KeywordTok{as.factor}\NormalTok{(myPixel}\OperatorTok{$}\NormalTok{day)   }\CommentTok{# convert day to a factor}
\NormalTok{list <-}\StringTok{ }\KeywordTok{rowSums}\NormalTok{(}\KeywordTok{table}\NormalTok{(myPixel}\OperatorTok{$}\NormalTok{Dog,myPixel}\OperatorTok{$}\NormalTok{dayFact))}\OperatorTok{==}\DecValTok{8}
\NormalTok{ist <-}\StringTok{ }\NormalTok{list[list}\OperatorTok{==}\OtherTok{TRUE}\NormalTok{] }\CommentTok{# Keep only those with 12 observations}
\NormalTok{list <-}\StringTok{ }\KeywordTok{as.numeric}\NormalTok{(}\KeywordTok{names}\NormalTok{(list)) }\CommentTok{# Extract the row indices}
\NormalTok{myPixel <-}\StringTok{ }\NormalTok{myPixel[myPixel}\OperatorTok{$}\NormalTok{Dog }\OperatorTok\StringTok{ }\NormalTok{list,] }\CommentTok{# Match against the data}

\CommentTok{#box plot to compare the tensions}
\KeywordTok{boxplot}\NormalTok{(pixel}\OperatorTok{~}\NormalTok{dayFact, }\DataTypeTok{data=}\NormalTok{myPixel,}
       \DataTypeTok{border=}\StringTok{"orange"}\NormalTok{, }
       \DataTypeTok{col=}\StringTok{"steelblue"}\NormalTok{,}
       \DataTypeTok{freq=}\OtherTok{FALSE}\NormalTok{,}
       \DataTypeTok{las=}\DecValTok{1}\NormalTok{, }
       \CommentTok{# breaks=10,}
       \CommentTok{# notch = TRUE,}
       \DataTypeTok{horizontal =} \OtherTok{FALSE}\NormalTok{ ,}\DataTypeTok{main=}\StringTok{" Balanced data - Distribution of pixel vs day"}\NormalTok{)}
\end{Highlighting}
\end{Shaded}

\includegraphics{HW10_Rajendiran_files/figure-latex/unnamed-chunk-7-1.pdf}

\begin{Shaded}
\begin{Highlighting}[]
\CommentTok{# Repeated measures ANOVA}
\NormalTok{aovpixelOut <-}\StringTok{ }\KeywordTok{aov}\NormalTok{(pixel }\OperatorTok{~}\StringTok{ }\NormalTok{dayFact}\OperatorTok{+}\KeywordTok{Error}\NormalTok{(Dog), }\DataTypeTok{data=}\NormalTok{myPixel)}
\KeywordTok{summary}\NormalTok{(aovpixelOut)}
\end{Highlighting}
\end{Shaded}

\begin{verbatim}
## 
## Error: Dog
##           Df Sum Sq Mean Sq F value Pr(>F)
## dayFact    2   2887    1444   0.208  0.817
## Residuals  7  48506    6929               
## 
## Error: Within
##           Df Sum Sq Mean Sq F value Pr(>F)  
## dayFact    3   1572   523.9   2.993 0.0383 *
## Residuals 57   9978   175.1                 
## ---
## Signif. codes:  0 '***' 0.001 '**' 0.01 '*' 0.05 '.' 0.1 ' ' 1
\end{verbatim}

\emph{as the starting point for cleaning up the data set.}

\hypertarget{chapter-11-exercise-5}{%
\section{Chapter 11, Exercise 5}\label{chapter-11-exercise-5}}

\emph{Given that the built-in AirPassengers data set (see ``?
AirPassengers'' for documentation) has a substantial growth trend, use
diff() to create a differenced data set. (1 pt) Use plot() to examine
and interpret the results of differencing. Use cpt.var() to find the
change point in the variability of the differenced time series. (1 pt)
Plot the result and describe in your own words what the change point
signifies. (1 pt)}

\begin{Shaded}
\begin{Highlighting}[]
\KeywordTok{EnsurePackage}\NormalTok{(}\StringTok{"changepoint"}\NormalTok{)}
\end{Highlighting}
\end{Shaded}

\begin{verbatim}
## Loading required package: changepoint
\end{verbatim}

\begin{verbatim}
## Loading required package: zoo
\end{verbatim}

\begin{verbatim}
## 
## Attaching package: 'zoo'
\end{verbatim}

\begin{verbatim}
## The following objects are masked from 'package:base':
## 
##     as.Date, as.Date.numeric
\end{verbatim}

\begin{verbatim}
## Successfully loaded changepoint package version 2.2.2
##  NOTE: Predefined penalty values changed in version 2.2.  Previous penalty values with a postfix 1 i.e. SIC1 are now without i.e. SIC and previous penalties without a postfix i.e. SIC are now with a postfix 0 i.e. SIC0. See NEWS and help files for further details.
\end{verbatim}

\begin{Shaded}
\begin{Highlighting}[]
\KeywordTok{EnsurePackage}\NormalTok{(}\StringTok{"tseries"}\NormalTok{)}
\end{Highlighting}
\end{Shaded}

\begin{verbatim}
## Loading required package: tseries
\end{verbatim}

\begin{verbatim}
## Registered S3 method overwritten by 'quantmod':
##   method            from
##   as.zoo.data.frame zoo
\end{verbatim}

\begin{Shaded}
\begin{Highlighting}[]
\NormalTok{? AirPassengers}
\CommentTok{# Monthly Airline Passenger Numbers 1949-1960}
\CommentTok{# Description: The classic Box & Jenkins airline data. }
\CommentTok{# Monthly totals of international airline passengers, 1949 to 1960.}
\CommentTok{# Usage : AirPassengers}
\CommentTok{# Format: A monthly time series, in thousands.}
\KeywordTok{summary}\NormalTok{(AirPassengers)}
\end{Highlighting}
\end{Shaded}

\begin{verbatim}
##    Min. 1st Qu.  Median    Mean 3rd Qu.    Max. 
##   104.0   180.0   265.5   280.3   360.5   622.0
\end{verbatim}

\begin{Shaded}
\begin{Highlighting}[]
\KeywordTok{str}\NormalTok{(AirPassengers) }\CommentTok{# Time-Series [1:144] from 1949 to 1961}
\end{Highlighting}
\end{Shaded}

\begin{verbatim}
##  Time-Series [1:144] from 1949 to 1961: 112 118 132 129 121 135 148 148 136 119 ...
\end{verbatim}

\begin{Shaded}
\begin{Highlighting}[]
\CommentTok{# View(AirPassengers)}
\end{Highlighting}
\end{Shaded}

\begin{Shaded}
\begin{Highlighting}[]
\CommentTok{# AirPassengers undifferential plot}
\KeywordTok{plot}\NormalTok{(AirPassengers,}\DataTypeTok{col=}\StringTok{"blue"}\NormalTok{,}\DataTypeTok{main=}\StringTok{"AirPassengers undifferential plot"}\NormalTok{)}
\end{Highlighting}
\end{Shaded}

\includegraphics{HW10_Rajendiran_files/figure-latex/unnamed-chunk-10-1.pdf}

\begin{Shaded}
\begin{Highlighting}[]
\CommentTok{# differenced data set}
\NormalTok{diffAirPassengers <-}\StringTok{ }\KeywordTok{diff}\NormalTok{(AirPassengers)}
\KeywordTok{plot}\NormalTok{(diffAirPassengers,}\DataTypeTok{col=}\StringTok{"steelblue"}\NormalTok{,}\DataTypeTok{main=}\StringTok{"AirPassengers differential time series"}\NormalTok{)}
\end{Highlighting}
\end{Shaded}

\includegraphics{HW10_Rajendiran_files/figure-latex/unnamed-chunk-11-1.pdf}

\begin{Shaded}
\begin{Highlighting}[]
\CommentTok{# change point in the variability of the differenced time series}
\NormalTok{dAPcp <-}\StringTok{ }\KeywordTok{cpt.var}\NormalTok{(diffAirPassengers)}
\KeywordTok{plot}\NormalTok{(dAPcp,}\DataTypeTok{cpt.col=}\StringTok{"orange"}\NormalTok{,}\DataTypeTok{cpt.width=}\DecValTok{5}\NormalTok{,}\DataTypeTok{col=}\StringTok{"steelblue"}
\NormalTok{     ,}\DataTypeTok{main=}\StringTok{"change point variance on differenced time series"}\NormalTok{)}
\end{Highlighting}
\end{Shaded}

\includegraphics{HW10_Rajendiran_files/figure-latex/unnamed-chunk-12-1.pdf}

\begin{Shaded}
\begin{Highlighting}[]
\KeywordTok{summary}\NormalTok{(dAPcp)}
\end{Highlighting}
\end{Shaded}

\begin{verbatim}
## Created Using changepoint version 2.2.2 
## Changepoint type      : Change in variance 
## Method of analysis    : AMOC 
## Test Statistic  : Normal 
## Type of penalty       : MBIC with value, 14.88853 
## Minimum Segment Length : 2 
## Maximum no. of cpts   : 1 
## Changepoint Locations : 76
\end{verbatim}

\hypertarget{chapter-11-exercise-6}{%
\section{Chapter 11, Exercise 6}\label{chapter-11-exercise-6}}

\emph{Use cpt.mean() on the undifferenced AirPassengers time series. (1
pt) Plot and interpret the results. Compare the change point of the mean
that you uncovered in this case to the change point in the variance that
you uncovered in Exercise 5. What do these change points suggest about
the history of air travel? (1 pt)}

\begin{Shaded}
\begin{Highlighting}[]
\CommentTok{# change point in the variability of the differenced time series}
\NormalTok{APcpm <-}\StringTok{ }\KeywordTok{cpt.mean}\NormalTok{(AirPassengers)}
\KeywordTok{plot}\NormalTok{(APcpm,}\DataTypeTok{cpt.col=}\StringTok{"steelblue"}\NormalTok{,}\DataTypeTok{cpt.width=}\DecValTok{5}\NormalTok{,}\DataTypeTok{col=}\StringTok{"brown"}
\NormalTok{     ,}\DataTypeTok{main=}\StringTok{"change point mean on undifferenced time series"}\NormalTok{)}
\end{Highlighting}
\end{Shaded}

\includegraphics{HW10_Rajendiran_files/figure-latex/unnamed-chunk-13-1.pdf}

\begin{Shaded}
\begin{Highlighting}[]
\KeywordTok{summary}\NormalTok{(APcpm)}
\end{Highlighting}
\end{Shaded}

\begin{verbatim}
## Created Using changepoint version 2.2.2 
## Changepoint type      : Change in mean 
## Method of analysis    : AMOC 
## Test Statistic  : Normal 
## Type of penalty       : MBIC with value, 14.90944 
## Minimum Segment Length : 1 
## Maximum no. of cpts   : 1 
## Changepoint Locations : 77
\end{verbatim}

\hypertarget{chapter-11-exercise-7}{%
\section{Chapter 11, Exercise 7}\label{chapter-11-exercise-7}}

\emph{Find historical information about air travel on the Internet
and/or in reference materials that sheds light on the results from
Exercises 5 and 6. Write a mini-article (less than 250 words) that
interprets your statistical findings from Exercises 5 and 6 in the
context of the historical information you found. (1 pt)}

\hypertarget{chapter-11-exercise-8}{%
\section{Chapter 11, Exercise 8}\label{chapter-11-exercise-8}}

\emph{Use bcp() on the AirPassengers time series. (1 pt) Plot and
interpret the results. Make sure to contrast these results with those
from Exercise 6. (1 pt)}

\begin{Shaded}
\begin{Highlighting}[]
\KeywordTok{EnsurePackage}\NormalTok{(}\StringTok{"bcp"}\NormalTok{)}
\end{Highlighting}
\end{Shaded}

\begin{verbatim}
## Loading required package: bcp
\end{verbatim}

\begin{verbatim}
## Loading required package: grid
\end{verbatim}

\begin{Shaded}
\begin{Highlighting}[]
\CommentTok{# Bayesian Change point analysis}
\NormalTok{bAPcpm <-}\StringTok{ }\KeywordTok{bcp}\NormalTok{(}\KeywordTok{as.vector}\NormalTok{(AirPassengers))}
\KeywordTok{plot}\NormalTok{(bAPcpm)}
\end{Highlighting}
\end{Shaded}

\includegraphics{HW10_Rajendiran_files/figure-latex/unnamed-chunk-14-1.pdf}

\begin{Shaded}
\begin{Highlighting}[]
\CommentTok{# summary(bAPcpm)}
\end{Highlighting}
\end{Shaded}

\begin{Shaded}
\begin{Highlighting}[]
\KeywordTok{plot}\NormalTok{(bAPcpm}\OperatorTok{$}\NormalTok{posterior.prob }\OperatorTok{>}\NormalTok{.}\DecValTok{95}\NormalTok{,}\DataTypeTok{col=}\StringTok{"blue"}\NormalTok{)}
\end{Highlighting}
\end{Shaded}

\includegraphics{HW10_Rajendiran_files/figure-latex/unnamed-chunk-15-1.pdf}

\end{document}
