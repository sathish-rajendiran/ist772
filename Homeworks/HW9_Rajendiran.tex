\PassOptionsToPackage{unicode=true}{hyperref} % options for packages loaded elsewhere
\PassOptionsToPackage{hyphens}{url}
%
\documentclass[]{article}
\usepackage{lmodern}
\usepackage{amssymb,amsmath}
\usepackage{ifxetex,ifluatex}
\usepackage{fixltx2e} % provides \textsubscript
\ifnum 0\ifxetex 1\fi\ifluatex 1\fi=0 % if pdftex
  \usepackage[T1]{fontenc}
  \usepackage[utf8]{inputenc}
  \usepackage{textcomp} % provides euro and other symbols
\else % if luatex or xelatex
  \usepackage{unicode-math}
  \defaultfontfeatures{Ligatures=TeX,Scale=MatchLowercase}
\fi
% use upquote if available, for straight quotes in verbatim environments
\IfFileExists{upquote.sty}{\usepackage{upquote}}{}
% use microtype if available
\IfFileExists{microtype.sty}{%
\usepackage[]{microtype}
\UseMicrotypeSet[protrusion]{basicmath} % disable protrusion for tt fonts
}{}
\IfFileExists{parskip.sty}{%
\usepackage{parskip}
}{% else
\setlength{\parindent}{0pt}
\setlength{\parskip}{6pt plus 2pt minus 1pt}
}
\usepackage{hyperref}
\hypersetup{
            pdftitle={IST772-- Problem Set 9},
            pdfauthor={Sathish Kumar Rajendiran},
            pdfborder={0 0 0},
            breaklinks=true}
\urlstyle{same}  % don't use monospace font for urls
\usepackage[margin=1in]{geometry}
\usepackage{color}
\usepackage{fancyvrb}
\newcommand{\VerbBar}{|}
\newcommand{\VERB}{\Verb[commandchars=\\\{\}]}
\DefineVerbatimEnvironment{Highlighting}{Verbatim}{commandchars=\\\{\}}
% Add ',fontsize=\small' for more characters per line
\usepackage{framed}
\definecolor{shadecolor}{RGB}{248,248,248}
\newenvironment{Shaded}{\begin{snugshade}}{\end{snugshade}}
\newcommand{\AlertTok}[1]{\textcolor[rgb]{0.94,0.16,0.16}{#1}}
\newcommand{\AnnotationTok}[1]{\textcolor[rgb]{0.56,0.35,0.01}{\textbf{\textit{#1}}}}
\newcommand{\AttributeTok}[1]{\textcolor[rgb]{0.77,0.63,0.00}{#1}}
\newcommand{\BaseNTok}[1]{\textcolor[rgb]{0.00,0.00,0.81}{#1}}
\newcommand{\BuiltInTok}[1]{#1}
\newcommand{\CharTok}[1]{\textcolor[rgb]{0.31,0.60,0.02}{#1}}
\newcommand{\CommentTok}[1]{\textcolor[rgb]{0.56,0.35,0.01}{\textit{#1}}}
\newcommand{\CommentVarTok}[1]{\textcolor[rgb]{0.56,0.35,0.01}{\textbf{\textit{#1}}}}
\newcommand{\ConstantTok}[1]{\textcolor[rgb]{0.00,0.00,0.00}{#1}}
\newcommand{\ControlFlowTok}[1]{\textcolor[rgb]{0.13,0.29,0.53}{\textbf{#1}}}
\newcommand{\DataTypeTok}[1]{\textcolor[rgb]{0.13,0.29,0.53}{#1}}
\newcommand{\DecValTok}[1]{\textcolor[rgb]{0.00,0.00,0.81}{#1}}
\newcommand{\DocumentationTok}[1]{\textcolor[rgb]{0.56,0.35,0.01}{\textbf{\textit{#1}}}}
\newcommand{\ErrorTok}[1]{\textcolor[rgb]{0.64,0.00,0.00}{\textbf{#1}}}
\newcommand{\ExtensionTok}[1]{#1}
\newcommand{\FloatTok}[1]{\textcolor[rgb]{0.00,0.00,0.81}{#1}}
\newcommand{\FunctionTok}[1]{\textcolor[rgb]{0.00,0.00,0.00}{#1}}
\newcommand{\ImportTok}[1]{#1}
\newcommand{\InformationTok}[1]{\textcolor[rgb]{0.56,0.35,0.01}{\textbf{\textit{#1}}}}
\newcommand{\KeywordTok}[1]{\textcolor[rgb]{0.13,0.29,0.53}{\textbf{#1}}}
\newcommand{\NormalTok}[1]{#1}
\newcommand{\OperatorTok}[1]{\textcolor[rgb]{0.81,0.36,0.00}{\textbf{#1}}}
\newcommand{\OtherTok}[1]{\textcolor[rgb]{0.56,0.35,0.01}{#1}}
\newcommand{\PreprocessorTok}[1]{\textcolor[rgb]{0.56,0.35,0.01}{\textit{#1}}}
\newcommand{\RegionMarkerTok}[1]{#1}
\newcommand{\SpecialCharTok}[1]{\textcolor[rgb]{0.00,0.00,0.00}{#1}}
\newcommand{\SpecialStringTok}[1]{\textcolor[rgb]{0.31,0.60,0.02}{#1}}
\newcommand{\StringTok}[1]{\textcolor[rgb]{0.31,0.60,0.02}{#1}}
\newcommand{\VariableTok}[1]{\textcolor[rgb]{0.00,0.00,0.00}{#1}}
\newcommand{\VerbatimStringTok}[1]{\textcolor[rgb]{0.31,0.60,0.02}{#1}}
\newcommand{\WarningTok}[1]{\textcolor[rgb]{0.56,0.35,0.01}{\textbf{\textit{#1}}}}
\usepackage{graphicx,grffile}
\makeatletter
\def\maxwidth{\ifdim\Gin@nat@width>\linewidth\linewidth\else\Gin@nat@width\fi}
\def\maxheight{\ifdim\Gin@nat@height>\textheight\textheight\else\Gin@nat@height\fi}
\makeatother
% Scale images if necessary, so that they will not overflow the page
% margins by default, and it is still possible to overwrite the defaults
% using explicit options in \includegraphics[width, height, ...]{}
\setkeys{Gin}{width=\maxwidth,height=\maxheight,keepaspectratio}
\setlength{\emergencystretch}{3em}  % prevent overfull lines
\providecommand{\tightlist}{%
  \setlength{\itemsep}{0pt}\setlength{\parskip}{0pt}}
\setcounter{secnumdepth}{0}
% Redefines (sub)paragraphs to behave more like sections
\ifx\paragraph\undefined\else
\let\oldparagraph\paragraph
\renewcommand{\paragraph}[1]{\oldparagraph{#1}\mbox{}}
\fi
\ifx\subparagraph\undefined\else
\let\oldsubparagraph\subparagraph
\renewcommand{\subparagraph}[1]{\oldsubparagraph{#1}\mbox{}}
\fi

% set default figure placement to htbp
\makeatletter
\def\fps@figure{htbp}
\makeatother

\usepackage{booktabs}
\usepackage{longtable}
\usepackage{array}
\usepackage{multirow}
\usepackage{wrapfig}
\usepackage{float}
\usepackage{colortbl}
\usepackage{pdflscape}
\usepackage{tabu}
\usepackage{threeparttable}
\usepackage{threeparttablex}
\usepackage[normalem]{ulem}
\usepackage{makecell}
\usepackage{xcolor}

\title{IST772-- Problem Set 9}
\author{Sathish Kumar Rajendiran}
\date{}

\begin{document}
\maketitle

Attribution statement: 1. I did this homework by myself, with help from
the book and the professor.

\begin{Shaded}
\begin{Highlighting}[]
\CommentTok{# import libraries }
\CommentTok{#create a function to ensure the libraries are imported}
\NormalTok{EnsurePackage <-}\StringTok{ }\ControlFlowTok{function}\NormalTok{(x)\{}
\NormalTok{  x <-}\StringTok{ }\KeywordTok{as.character}\NormalTok{(x)}
    \ControlFlowTok{if}\NormalTok{ (}\OperatorTok{!}\KeywordTok{require}\NormalTok{(x,}\DataTypeTok{character.only =} \OtherTok{TRUE}\NormalTok{))\{}
      \KeywordTok{install.packages}\NormalTok{(}\DataTypeTok{pkgs=}\NormalTok{x, }\DataTypeTok{repos =} \StringTok{"http://cran.us.r-project.org"}\NormalTok{)}
      \KeywordTok{require}\NormalTok{(x, }\DataTypeTok{character.only =} \OtherTok{TRUE}\NormalTok{)}
\NormalTok{    \}}
\NormalTok{  \}}
\end{Highlighting}
\end{Shaded}

\hypertarget{chapter-10-exercise-1}{%
\section{Chapter 10, Exercise 1}\label{chapter-10-exercise-1}}

\emph{The data sets package in R contains a small data set called swiss
that contains n = 47 observations of socio-economic indicators for each
of 47 French-speaking provinces of Switzerland at about 1888. Use
``?swiss'' to display help about the data set. All the data in this data
set are metric, but one, Catholic, shows a very bimodal distribution. We
can dichotomize this variable to create binary variable as follows:}

\begin{Shaded}
\begin{Highlighting}[]
\NormalTok{?swiss }
\CommentTok{# Swiss Fertility and Socioeconomic Indicators (1888) Data}
\CommentTok{# Standardized fertility measure and socio-economic indicators for each of 47 French-speaking provinces of Switzerland at about 1888.}

\CommentTok{# A data frame with 47 observations on 6 variables, each of which is in percent, i.e., in [0, 100].}
\CommentTok{# [,1]  Fertility   Ig, ‘common standardized fertility measure’}
\CommentTok{# [,2]  Agriculture % of males involved in agriculture as occupation}
\CommentTok{# [,3]  Examination % draftees receiving highest mark on army examination}
\CommentTok{# [,4]  Education   % education beyond primary school for draftees.}
\CommentTok{# [,5]  Catholic    % ‘catholic’ (as opposed to ‘protestant’).}
\CommentTok{# [,6]  Infant.Mortality    live births who live less than 1 year.}
\CommentTok{# All variables but ‘Fertility’ give proportions of the population.}

\KeywordTok{summary}\NormalTok{(swiss)}
\end{Highlighting}
\end{Shaded}

\begin{verbatim}
##    Fertility      Agriculture     Examination      Education    
##  Min.   :35.00   Min.   : 1.20   Min.   : 3.00   Min.   : 1.00  
##  1st Qu.:64.70   1st Qu.:35.90   1st Qu.:12.00   1st Qu.: 6.00  
##  Median :70.40   Median :54.10   Median :16.00   Median : 8.00  
##  Mean   :70.14   Mean   :50.66   Mean   :16.49   Mean   :10.98  
##  3rd Qu.:78.45   3rd Qu.:67.65   3rd Qu.:22.00   3rd Qu.:12.00  
##  Max.   :92.50   Max.   :89.70   Max.   :37.00   Max.   :53.00  
##     Catholic       Infant.Mortality
##  Min.   :  2.150   Min.   :10.80   
##  1st Qu.:  5.195   1st Qu.:18.15   
##  Median : 15.140   Median :20.00   
##  Mean   : 41.144   Mean   :19.94   
##  3rd Qu.: 93.125   3rd Qu.:21.70   
##  Max.   :100.000   Max.   :26.60
\end{verbatim}

\begin{Shaded}
\begin{Highlighting}[]
\KeywordTok{dim}\NormalTok{(swiss)}
\end{Highlighting}
\end{Shaded}

\begin{verbatim}
## [1] 47  6
\end{verbatim}

\begin{Shaded}
\begin{Highlighting}[]
\KeywordTok{View}\NormalTok{(swiss)}

\KeywordTok{str}\NormalTok{(swiss)}
\end{Highlighting}
\end{Shaded}

\begin{verbatim}
## 'data.frame':    47 obs. of  6 variables:
##  $ Fertility       : num  80.2 83.1 92.5 85.8 76.9 76.1 83.8 92.4 82.4 82.9 ...
##  $ Agriculture     : num  17 45.1 39.7 36.5 43.5 35.3 70.2 67.8 53.3 45.2 ...
##  $ Examination     : int  15 6 5 12 17 9 16 14 12 16 ...
##  $ Education       : int  12 9 5 7 15 7 7 8 7 13 ...
##  $ Catholic        : num  9.96 84.84 93.4 33.77 5.16 ...
##  $ Infant.Mortality: num  22.2 22.2 20.2 20.3 20.6 26.6 23.6 24.9 21 24.4 ...
\end{verbatim}

\begin{Shaded}
\begin{Highlighting}[]
\CommentTok{# define variables}
\NormalTok{Fertility <-}\StringTok{ }\NormalTok{swiss}\OperatorTok{$}\NormalTok{Fertility}
\NormalTok{Education <-}\StringTok{ }\NormalTok{swiss}\OperatorTok{$}\NormalTok{Education}
\NormalTok{Examination <-}\StringTok{ }\NormalTok{swiss}\OperatorTok{$}\NormalTok{Examination}
\NormalTok{Catholic <-}\StringTok{ }\NormalTok{swiss}\OperatorTok{$}\NormalTok{Catholic}
\NormalTok{Infant <-}\StringTok{ }\NormalTok{swiss}\OperatorTok{$}\NormalTok{Infant}
\NormalTok{Agriculture <-}\StringTok{ }\NormalTok{swiss}\OperatorTok{$}\NormalTok{Agriculture}

\KeywordTok{colnames}\NormalTok{(swiss)}
\end{Highlighting}
\end{Shaded}

\begin{verbatim}
## [1] "Fertility"        "Agriculture"      "Examination"      "Education"       
## [5] "Catholic"         "Infant.Mortality"
\end{verbatim}

\begin{Shaded}
\begin{Highlighting}[]
\CommentTok{# load the necessary library for further processing...}
\KeywordTok{EnsurePackage}\NormalTok{(}\StringTok{"tidyverse"}\NormalTok{)}
\end{Highlighting}
\end{Shaded}

\begin{verbatim}
## Loading required package: tidyverse
\end{verbatim}

\begin{verbatim}
## -- Attaching packages --------------------------------------- tidyverse 1.3.0 --
\end{verbatim}

\begin{verbatim}
## v ggplot2 3.3.2     v purrr   0.3.4
## v tibble  3.0.1     v dplyr   1.0.0
## v tidyr   1.0.2     v stringr 1.4.0
## v readr   1.3.1     v forcats 0.5.0
\end{verbatim}

\begin{verbatim}
## -- Conflicts ------------------------------------------ tidyverse_conflicts() --
## x dplyr::filter() masks stats::filter()
## x dplyr::lag()    masks stats::lag()
\end{verbatim}

\begin{Shaded}
\begin{Highlighting}[]
\NormalTok{swiss }\OperatorTok\StringTok{ }
\StringTok{  }\KeywordTok{pivot_longer}\NormalTok{(}\DataTypeTok{cols=} \OperatorTok{-}\NormalTok{Catholic, }\DataTypeTok{names_to=}\StringTok{"variable"}\NormalTok{, }\DataTypeTok{values_to=}\StringTok{"value"}\NormalTok{, }\DataTypeTok{values_drop_na =} \OtherTok{TRUE}\NormalTok{) }\OperatorTok\StringTok{ }
\StringTok{  }\KeywordTok{ggplot}\NormalTok{(}\KeywordTok{aes}\NormalTok{(}\DataTypeTok{x=}\NormalTok{variable, }\DataTypeTok{y=}\NormalTok{value)) }\OperatorTok{+}\StringTok{ }\KeywordTok{geom_violin}\NormalTok{() }\OperatorTok{+}\StringTok{ }\KeywordTok{facet_wrap}\NormalTok{( }\OperatorTok{~}\StringTok{ }\NormalTok{variable, }\DataTypeTok{scales=}\StringTok{"free"}\NormalTok{)}
\end{Highlighting}
\end{Shaded}

\includegraphics{HW9_Rajendiran_files/figure-latex/unnamed-chunk-3-1.pdf}

\begin{Shaded}
\begin{Highlighting}[]
\NormalTok{swiss}\OperatorTok{$}\NormalTok{Catholic.b <-}\StringTok{ }\KeywordTok{as.integer}\NormalTok{(swiss}\OperatorTok{$}\NormalTok{Catholic }\OperatorTok{>}\StringTok{ }\DecValTok{60}\NormalTok{) }\CommentTok{# 60 looks like a gap in the histogram}
\KeywordTok{table}\NormalTok{(swiss}\OperatorTok{$}\NormalTok{Catholic.b)}
\end{Highlighting}
\end{Shaded}

\begin{verbatim}
## 
##  0  1 
## 31 16
\end{verbatim}

\begin{Shaded}
\begin{Highlighting}[]
\CommentTok{# Corellation between Agriculture and Fertility}
\KeywordTok{plot}\NormalTok{(Agriculture, Fertility}
\NormalTok{     ,}\DataTypeTok{main=}\StringTok{"Corellation between Agriculture and Fertility"}
\NormalTok{     ,}\DataTypeTok{col=}\StringTok{"red"}\NormalTok{)}
\end{Highlighting}
\end{Shaded}

\includegraphics{HW9_Rajendiran_files/figure-latex/unnamed-chunk-5-1.pdf}

\emph{Use logistic regression to predict Catholic.b, using two metric
variables in the data set, Fertility and Agriculture (1 pt). Run any
necessary diagnostics. (1 pt) Interpret the resulting null hypothesis
significance tests. (1 pt)}

\hypertarget{question-1-logistic-regression-analysis-on-swiss-data}{%
\section{1) Question 1: Logistic Regression Analysis on Swiss
data}\label{question-1-logistic-regression-analysis-on-swiss-data}}

\begin{itemize}
\tightlist
\item
  glm() \textbar{} Dependent variable (Catholic.b) \& Fertility and
  Agriculture as Predictors/Independent variables with binomial(link =
  ``logit'') link function

  \begin{itemize}
  \tightlist
  \item
    glm(Catholic.b \textasciitilde{} Fertility +
    Agriculture,data=swiss,family = binomial(link = ``logit'')) is
    stored on a variable ``swiss.glm''
  \item
    above code runs a logistic regression analysis on Catholic.b as the
    dependent variable and Fertility and Agriculture as the predictors
    from ``swiss'' dataset.
  \end{itemize}
\item
  Diagnostics:

  \begin{itemize}
  \tightlist
  \item
    From the glm(Catholic.b \textasciitilde{} Fertility +
    Agriculture,data=swiss,family = binomial(link = ``logit'')) ; this
    formula predicts ``Catholic.b'' values from Fertility and
    Agriculture combined from the dataset ``swiss''
  \item
    Procedure glm() is similar to lm() procedure and expects the
    dependent variable (variable, in question for prediction) first.
    Followed by, independent/prdictor variable and a link function. It
    can have number of predictor variables follows after
    ``\textasciitilde{}'' symbol. ``.'' after ``\textasciitilde{}''
    means - it expects to include all the remaining variables from the
    dataset. In this case, we are only trying to predict ``fertility''
    rate based on ``Education'' \& ``Agriculture'' variable.
  \item
    In addition, ``data=swiss'' implies what is the sample/population
    the prediction is run against from the observations it contains.
  \item
    Successul execution of the glm() procedure provides results as shown
    below.

    \begin{itemize}
    \tightlist
    \item
      call
    \item
      Deviance Residuals
    \item
      Coefficients
    \item
      Significant codes
    \item
      Null and Residual deviance with degrees of freedom
    \item
      Number of Fisher Scoring iterations
    \end{itemize}
  \item
    Chi-Square analysis on logistic regression

    \begin{itemize}
    \tightlist
    \item
      anova(swiss.glm, test=``Chisq'') -performs chi-square analysis on
      the glm() output
    \end{itemize}
  \item
    Convert the log odds for the coefficient on the predictor into
    regular odds

    \begin{itemize}
    \tightlist
    \item
      exp(coef(glmOut)) - converts log odds into regular odds
    \end{itemize}
  \item
    Null hypothesis
  \end{itemize}
\item
  Results:

  \begin{itemize}
  \tightlist
  \item
    Once the glm() procedure executes successfuly, it returns various
    data points as an outcome. The first two lines defines the model, we
    wanted.

    \begin{itemize}
    \tightlist
    \item
      glm(formula = Catholic.b \textasciitilde{} Fertility +
      Agriculture, family = binomial(link = ``logit''), data = swiss)
    \item
      By specifying binomial() - it invokes the inverse logit or
      logistic function as the basis for fitting the X variables to the
      Y variable.
    \end{itemize}
  \item
    Next, Summary of residuals that gives an overview of errors of
    prediction.

    \begin{itemize}
    \tightlist
    \item
      With min as -1.50316 and max of 2.39654 shows, distribution of
      residuals between -ve to positive almost spreading equally on both
      sides, with Median almost 0 (-0.01404).
    \item
      It seems the residuals are symmetrically distributed.
    \item
      hist(residuals(swiss.glm)) suggests the same.
    \end{itemize}
  \item
    Coefficients shows the key results.

    \begin{itemize}
    \item
      Intercept is at -34.07275.
    \item
      Slopes for Education variable is 0.35010 and Agriculture is
      0.13078 are way off from the intercept or B-weights. These
      coefficients define the logrithm of the odds of the Y variable.
    \item
      Std.Errors around the estimates of slope and intercept shows the
      estimated sampling distribution around these point estimates.
    \item
      z-value shows the student's t-test of the null hypothesis test
      that each estimated coefficients is equal to zero.
    \item
      two asterik (*) indicates the significance level of alpha at p
      \textless{} 0.01.
    \item
      one asterik (*) indicates that the significance level of alpha
      level is 0.05.
    \item
      With above p-value; * Fertility has the strong coefficient value
      as 0.00293 far less than the test of significance (p \textless{}
      0.01); * Agriculture has p-value as 0.01384 far less than the test
      of significance (p \textless{} 0.05) This shows that , both
      ``Agriculture'' and ``Fertility'' are both statistically
      significant.
    \item
      Null Hypothesis * Null hypothesis is that the log odds of
      catholic.b is equal to 0 in the population. Since the log odds of
      Fertility and Education both are statistically significant and
      less than thier respective alpha levels - we can reject the null
      hypothesis.
    \item
      In addition, the conversion from log odds to regular output
      (exp(coef(glmOut))), (Intercept) Fertility Agriculture
      0.000000000000001593646 1.419211326551196750145
      1.139719619964769448117 From above its is infered that 1.419:1 on
      fertility and 1.139:1 on Agriculture to likely to claim catholic.b
    \item
      The first Chi-Square model compares three nested models.

      \begin{itemize}
      \tightlist
      \item
        anova(swiss.glm, test=``Chisq'') - includes both predictors and
        tests the level of significance on these predictors. It confirms
        that both predictors with 0.0000009474 and 0.0001207 is far less
        the the p value (0.001) and they are statistically significant.
      \item
        24.032 is the chi-square value ( residual deviance from top line
        - residual deviance from 2nd line) 60.284 - 36.252 = 24.032 is
        tested for significance on one degree of freedom.
      \end{itemize}
    \end{itemize}
  \end{itemize}
\end{itemize}

\hypertarget{please-find-more-details-from-the-r-code-below}{%
\section{Please find more details from the R code
below,}\label{please-find-more-details-from-the-r-code-below}}

\begin{Shaded}
\begin{Highlighting}[]
\KeywordTok{options}\NormalTok{(}\DataTypeTok{scipen=}\DecValTok{999}\NormalTok{)  }\CommentTok{# turn-off scientific notation like 1e+48}
\NormalTok{swiss.glm <-}\StringTok{ }\KeywordTok{glm}\NormalTok{(Catholic.b }\OperatorTok{~}\StringTok{ }\NormalTok{Fertility }\OperatorTok{+}\StringTok{ }\NormalTok{Agriculture,}\DataTypeTok{data=}\NormalTok{swiss,}\DataTypeTok{family =} \KeywordTok{binomial}\NormalTok{(}\DataTypeTok{link =} \StringTok{"logit"}\NormalTok{)) }
\KeywordTok{summary}\NormalTok{(swiss.glm)}
\end{Highlighting}
\end{Shaded}

\begin{verbatim}
## 
## Call:
## glm(formula = Catholic.b ~ Fertility + Agriculture, family = binomial(link = "logit"), 
##     data = swiss)
## 
## Deviance Residuals: 
##      Min        1Q    Median        3Q       Max  
## -1.50316  -0.24820  -0.01404   0.14290   2.39654  
## 
## Coefficients:
##              Estimate Std. Error z value Pr(>|z|)   
## (Intercept) -34.07275   11.31906  -3.010  0.00261 **
## Fertility     0.35010    0.11768   2.975  0.00293 **
## Agriculture   0.13078    0.05314   2.461  0.01384 * 
## ---
## Signif. codes:  0 '***' 0.001 '**' 0.01 '*' 0.05 '.' 0.1 ' ' 1
## 
## (Dispersion parameter for binomial family taken to be 1)
## 
##     Null deviance: 60.284  on 46  degrees of freedom
## Residual deviance: 21.470  on 44  degrees of freedom
## AIC: 27.47
## 
## Number of Fisher Scoring iterations: 8
\end{verbatim}

\begin{Shaded}
\begin{Highlighting}[]
\CommentTok{#Chi-Square analysis on logistic regression}
\KeywordTok{anova}\NormalTok{(swiss.glm, }\DataTypeTok{test=}\StringTok{"Chisq"}\NormalTok{)}
\end{Highlighting}
\end{Shaded}

\begin{verbatim}
## Analysis of Deviance Table
## 
## Model: binomial, link: logit
## 
## Response: Catholic.b
## 
## Terms added sequentially (first to last)
## 
## 
##             Df Deviance Resid. Df Resid. Dev     Pr(>Chi)    
## NULL                           46     60.284                 
## Fertility    1   24.032        45     36.252 0.0000009474 ***
## Agriculture  1   14.781        44     21.470    0.0001207 ***
## ---
## Signif. codes:  0 '***' 0.001 '**' 0.01 '*' 0.05 '.' 0.1 ' ' 1
\end{verbatim}

\begin{Shaded}
\begin{Highlighting}[]
\CommentTok{# Convert the log odds for the coefficient on the predictor into regular odds}
\KeywordTok{exp}\NormalTok{(}\KeywordTok{coef}\NormalTok{(swiss.glm))}
\end{Highlighting}
\end{Shaded}

\begin{verbatim}
##             (Intercept)               Fertility             Agriculture 
## 0.000000000000001593646 1.419211326551196750145 1.139719619964769448117
\end{verbatim}

\begin{Shaded}
\begin{Highlighting}[]
\KeywordTok{plot}\NormalTok{(swiss.glm)}
\end{Highlighting}
\end{Shaded}

\includegraphics{HW9_Rajendiran_files/figure-latex/unnamed-chunk-7-1.pdf}
\includegraphics{HW9_Rajendiran_files/figure-latex/unnamed-chunk-7-2.pdf}
\includegraphics{HW9_Rajendiran_files/figure-latex/unnamed-chunk-7-3.pdf}
\includegraphics{HW9_Rajendiran_files/figure-latex/unnamed-chunk-7-4.pdf}

\begin{Shaded}
\begin{Highlighting}[]
\KeywordTok{hist}\NormalTok{(}\KeywordTok{residuals}\NormalTok{(swiss.glm))}
\end{Highlighting}
\end{Shaded}

\includegraphics{HW9_Rajendiran_files/figure-latex/unnamed-chunk-8-1.pdf}

\hypertarget{chapter-10-exercise-5}{%
\section{Chapter 10, Exercise 5}\label{chapter-10-exercise-5}}

\emph{As noted in the chapter, the BaylorEdPsych add‐in package contains
a procedure for generating pseudo‐R‐squared values from the output of
the glm() procedure. Use the results of Exercise 1 to generate, report,
and interpret a Nagelkerke pseudo‐R‐squared value. You might also
examine the confusion matrix. (1 pt) }

\hypertarget{question-5-pseudorsquared-values}{%
\section{2) Question 5: Pseudo‐R‐squared
values}\label{question-5-pseudorsquared-values}}

\begin{itemize}
\tightlist
\item
  Pseudo-R-squared value is calculated as below,

  \begin{itemize}
  \tightlist
  \item
    PseudoR2(swiss.glm) - this formula outputs various coeffients
    including the Nagelkerke in scope.
  \end{itemize}
\item
  Nagelkerke - Interpretation

  \begin{itemize}
  \tightlist
  \item
    The Nagelkerke comes significanttly larger that than few the other
    values at 0.7778181

    \begin{itemize}
    \tightlist
    \item
      McFadden 0.6438474
    \item
      Adj.McFadden 0.5111418
    \item
      Cox.Snell 0.5621247
    \item
      McKelvey.Zavoina 0.9169675
    \item
      Effron 0.6902251
    \end{itemize}
  \item
    Nagelkerke R-squared value is interpreted as the propotion of
    variance in the outcome variable from earlier exercise
    ``Catholic.b'' accounted for the predictor variables ``Agriculture''
    and ``Fertility''.
  \item
    As only had a sample size of 47 observations, we were able to detect
    the much smaller effect on these predictor variables.
  \end{itemize}
\item
  In addtion, to conclude - lets explore on the confusion matrix as well

  \begin{itemize}
  \tightlist
  \item
    Its a contingency table that compares the observed outcome vs the
    predicted results
  \item
    table(round(predict(swiss.glm,type=``response'')),Catholic.b) -
    formula produces confusion matrix on the glm model output.
  \item
    Outcome is dichotimized to 0 and 1 signifying result towards
    Catholic.b or not.
  \item
    0 means Catholic.b as Yes and 1 as Catholic.b as No
  \item
    Overall accuracy is calculated as (29+14)/47 = 0.9148936 (
    \textasciitilde{}91.5\% accuracy with these predictor variables)
  \item
    91.5\% accuracy further suggesting the significance of the predictor
    variables.
  \end{itemize}
\end{itemize}

\hypertarget{please-find-more-details-from-the-r-code-below-1}{%
\section{Please find more details from the R code
below,}\label{please-find-more-details-from-the-r-code-below-1}}

\begin{Shaded}
\begin{Highlighting}[]
\CommentTok{# load the necessary library for further processing...}
\KeywordTok{EnsurePackage}\NormalTok{(}\StringTok{"BaylorEdPsych"}\NormalTok{)}
\end{Highlighting}
\end{Shaded}

\begin{verbatim}
## Loading required package: BaylorEdPsych
\end{verbatim}

\begin{Shaded}
\begin{Highlighting}[]
\KeywordTok{PseudoR2}\NormalTok{(swiss.glm) }\CommentTok{# generate pseudo‐R‐squared values}
\end{Highlighting}
\end{Shaded}

\begin{verbatim}
##         McFadden     Adj.McFadden        Cox.Snell       Nagelkerke 
##        0.6438474        0.5111418        0.5621247        0.7778181 
## McKelvey.Zavoina           Effron            Count        Adj.Count 
##        0.9169675        0.6902251        0.9148936        0.7500000 
##              AIC    Corrected.AIC 
##       27.4702414       28.0283810
\end{verbatim}

\begin{Shaded}
\begin{Highlighting}[]
\KeywordTok{cat}\NormalTok{(}\StringTok{"}\CharTok{\textbackslash{}n}\StringTok{confusion matrix:}\CharTok{\textbackslash{}n}\StringTok{"}\NormalTok{)}
\end{Highlighting}
\end{Shaded}

\begin{verbatim}
## 
## confusion matrix:
\end{verbatim}

\begin{Shaded}
\begin{Highlighting}[]
\CommentTok{#confusion matrix}
\KeywordTok{table}\NormalTok{(}\KeywordTok{round}\NormalTok{(}\KeywordTok{predict}\NormalTok{(swiss.glm,}\DataTypeTok{type=}\StringTok{"response"}\NormalTok{)),swiss}\OperatorTok{$}\NormalTok{Catholic.b)}
\end{Highlighting}
\end{Shaded}

\begin{verbatim}
##    
##      0  1
##   0 29  2
##   1  2 14
\end{verbatim}

\begin{Shaded}
\begin{Highlighting}[]
\CommentTok{# (29+14)/47}
\end{Highlighting}
\end{Shaded}

\hypertarget{chapter-10-exercise-6}{%
\section{Chapter 10, Exercise 6}\label{chapter-10-exercise-6}}

\emph{Continue the analysis of the Chile data set described in this
chapter. The data set is in the ``car'' package, so you will have to
install.packages() and library() that package first, and then use the
data(Chile) command to get access to the data set and ``? Chile'' to see
the documentation. Pay close attention to the transformations needed to
isolate cases with the Yes and No votes as shown in this chapter. Add a
new predictor, statusquo, into the model and remove the income variable.
Your new model specification should be vote \textasciitilde{} age +
statusquo. The statusquo variable is a rating that each respondent gave
indicating whether they preferred change or maintaining the status quo.
Conduct general linear model (1 pt + 1 pt) and Bayesian analysis on this
model (1 pt) and report and interpret all relevant results (1 pt).
Compare the AIC from this model to the AIC from the model that was
developed in the chapter (using income and age as predictors). }

\hypertarget{question-6-logistic-regression-analysis-on-chile-data}{%
\section{3) Question 6: Logistic Regression Analysis on Chile
data}\label{question-6-logistic-regression-analysis-on-chile-data}}

\begin{itemize}
\tightlist
\item
  Data Preprocessing:

  \begin{itemize}
  \tightlist
  \item
    Chile data is imped by enabling ``car'' package"
  \item
    ChileYN dataframe is created by having the values Y and N split from
    the observations with Chile\$vote==Y and N respectively
  \item
    removed ``income'' from this newly created dataframe
  \item
    missing values are removed
  \item
    Variable ChileYN\$vote is adjusted to Factor and the outcome is
    changed to numeric further to simply keep the values as 0 and 1; 0 =
    Vote and 1 = No vote (No)
  \end{itemize}
\item
  glm() on ChileYN dataset

  \begin{itemize}
  \tightlist
  \item
    glm(vote \textasciitilde{} age + statusquo,data=ChileYN,family =
    binomial(link = ``logit'')) is stored on a variable ``chile.glm''
  \item
    above code runs a logistic regression analysis on vote as the
    dependent variable and age and statusquo as the predictors from
    ``ChileYN'' dataset.
  \item
    ``binomial'' link function changes the output of the glm model to
    inverse logit or logistic regression
  \end{itemize}
\item
  Diagnostics:

  \begin{itemize}
  \tightlist
  \item
    From the glm(vote \textasciitilde{} age +
    statusquo,data=ChileYN,family = binomial(link = ``logit'')) ; this
    formula predicts ``vote'' values from age and statusquo combined
    from the dataset ``ChileYN''
  \item
    Procedure glm() is similar to lm() procedure and expects the
    dependent variable (variable, in question for prediction) first.
    Followed by, independent/prdictor variable and a link function. It
    can have number of predictor variables follows after
    ``\textasciitilde{}'' symbol. ``.'' after ``\textasciitilde{}''
    means - it expects to include all the remaining variables from the
    dataset. In this case, we are only trying to predict ``vote'' rate
    based on ``Age'' \& ``Statusquo'' variables.``binomial'' is the link
    function changes the output of the glm model to inverse logit or
    logistic regression
  \item
    In addition, ``data=ChileYN'' implies what is the sample/population
    the prediction is run against from the observations it contains.
  \item
    Successul execution of the glm() procedure provides results as shown
    below.

    \begin{itemize}
    \tightlist
    \item
      call
    \item
      Deviance Residuals
    \item
      Coefficients
    \item
      Significant codes
    \item
      Null and Residual deviance with degrees of freedom
    \item
      Number of Fisher Scoring iterations
    \end{itemize}
  \item
    Chi-Square analysis on logistic regression

    \begin{itemize}
    \tightlist
    \item
      anova(chile.glm, test=``Chisq'') -performs chi-square analysis on
      the glm() output
    \end{itemize}
  \item
    Convert the log odds for the coefficient on the predictor into
    regular odds

    \begin{itemize}
    \tightlist
    \item
      exp(coef(chile.glm)) - converts log odds into regular odds
    \end{itemize}
  \item
    Confusion matrix

    \begin{itemize}
    \tightlist
    \item
      table(round(predict(chile.glm,type=``response'')),ChileYN\$vote) -
      this formula creates confusion marix
    \end{itemize}
  \item
    Null hypothesis
  \item
    Run bayesian analysis on top the glm output

    \begin{itemize}
    \tightlist
    \item
      bayesLogitOut \textless{}- MCMClogit(formula = vote
      \textasciitilde{} age + statusquo, data = ChileYN) formula
      generates BayesLogit output
    \end{itemize}
  \item
    Confirm alternate hypothesis and its test of significance on the
    dependent variables.
  \item
    Run AIC procedure

    \begin{itemize}
    \tightlist
    \item
      stepAIC(chile.glm) formula generates AIC output
    \item
      Compare it against old AIC on age+income predictors
    \end{itemize}
  \end{itemize}
\item
  Results:

  \begin{itemize}
  \tightlist
  \item
    Once the glm() procedure executes successfuly, it returns various
    data points as an outcome. The first two lines defines the model, we
    wanted.

    \begin{itemize}
    \tightlist
    \item
      glm(formula = vote \textasciitilde{} age + statusquo, family =
      binomial(link = ``logit''), data = ChileYN)
    \item
      By specifying binomial() - it invokes the inverse logit or
      logistic function as the basis for fitting the X variables to the
      Y variable.
    \end{itemize}
  \item
    Next, Summary of residuals that gives an overview of errors of
    prediction.

    \begin{itemize}
    \tightlist
    \item
      With min as -3.2095 and max of 2.8789 shows, distribution of
      residuals between -ve to positive almost spreading equally on both
      sides, with Median almost 0 (-0.1840).
    \item
      It seems the residuals are symmetrically distributed.
    \item
      hist(residuals(chile.glm)) suggests the same.
    \end{itemize}
  \item
    Coefficients shows the key results.

    \begin{itemize}
    \tightlist
    \item
      Intercept is at -0.193759.
    \item
      Slopes for age variable is 0.011322 and statusquo is 3.174487 are
      way off from the intercept or B-weights. These coefficients define
      the logrithm of the odds of the Y variable.
    \item
      Std.Errors around the estimates of slope and intercept shows the
      estimated sampling distribution around these point estimates.
    \item
      z-value shows the student's t-test of the null hypothesis test
      that each estimated coefficients is equal to zero.
    \item
      With above p-value; * age has the weak coefficient value as
      0.011322 higher than the test of significance (p \textless{} 0.01)
      at 0.0972 * statusquo has p-value as 0.0000000000000002 far less
      than the test of significance (p \textless{} 0.001) This shows
      that , ``statusquo'' as statistically significant where age as not
      significant.
    \item
      Null Hypothesis * Null hypothesis is that the log odds of vote is
      equal to 0 in the population. Since the log odds of statusquo is
      statistically significant and less than thier respective alpha
      level - we can reject the null hypothesis. * However, age not
      having the stronger significance , we fail to reject null
      hypothesis with Age as predictor for Vote.
    \item
      In addition, the conversion from log odds to regular output
      (exp(coef(chile.glm))), (Intercept) age statusquo 0.8238564
      1.0113863 23.9145451 From above its is infered that 1.0113863:1 on
      age and 23.9145451:1 on statusquo to likely to claim vote. Showing
      stronger odds on statusquo to predict vote
    \item
      The first Chi-Square model compares three nested models.

      \begin{itemize}
      \tightlist
      \item
        anova(chile.glm, test=``Chisq'') - includes both predictors and
        tests the level of significance on these predictors.
      \item
        It confirms that both predictors with 0.000000004964 (age) and
        0.00000000000000022(statusquo) is far less the the p value
        (0.001) and they are statistically significant.
      \item
        34.2 is the chi-square value ( residual deviance from top line -
        residual deviance from 2nd line) 2360.29 - 2326.09 = 34.2 is
        tested for significance on one degree of freedom.
      \end{itemize}
    \end{itemize}
  \item
    Bayesian Logit * MCMCPack (MCMClogit) does the the bayesian
    estimation of logistic regression * bayesLogitOut \textless{}-
    MCMClogit(formula = vote \textasciitilde{} age + statusquo, data =
    ChileYN) formula is used to simulate Bayesian estimation * Wih
    sample size of 10,000 observations - population mean of the standard
    errors * With earlier pre-processing the depedent variable is
    converted into No=0 and Yes = 1. * Output contains the posterior
    distribution of parameters representing both intercept and
    cofficients on age and statusquo calibrated as log-odds. * Point
    extimates for the intercept and the coefficients are similar to
    outputs from the logistic regression. * 95\% HDI interval shows no
    overlap on age and statusquo variables. * 95\% HDI shows age as
    significant as it overlaps with 0. * more detailed analysis on the
    HDI is shown on the trace chart below and it captures extensive
    information about the alternative hypothesis for each of the
    coefficients being estimated. * density curve on age is spread
    across 0 and HDI lower bound (2.5\%) at -0.002005 and uppoer bound
    (97.5\%) at 0.02499. * density curve on statusquo shows HDI well
    over 0 with HDI lower bound (2.5\%) at 2.914442 and Upper bound
    (97.5\%) at 3.48698. favoring alternative hypothesis.
  \item
    AIC is calculated at 740.5207 on age + statusquo Step Df Deviance
    Resid. Df Resid. Dev AIC 1 1700 734.5207 740.5207
  \item
    AIC is at 2330.1 on age + income; with two step process Step Df
    Deviance Resid. Df Resid. Dev AIC 1 1700 2326.029 2332.029 2 -
    income 1 0.06212372 1701 2326.091 2330.091
  \end{itemize}
\end{itemize}

\hypertarget{please-find-more-details-from-the-r-code-below-2}{%
\section{Please find more details from the R code
below,}\label{please-find-more-details-from-the-r-code-below-2}}

\begin{Shaded}
\begin{Highlighting}[]
\KeywordTok{options}\NormalTok{(}\DataTypeTok{scipen=}\DecValTok{999}\NormalTok{)  }\CommentTok{# turn-off scientific notation like 1e+48}

\CommentTok{# load the necessary library for further processing...}
\KeywordTok{EnsurePackage}\NormalTok{(}\StringTok{"car"}\NormalTok{) }\CommentTok{# Regression helper package: Chile data}
\end{Highlighting}
\end{Shaded}

\begin{verbatim}
## Loading required package: car
\end{verbatim}

\begin{verbatim}
## Loading required package: carData
\end{verbatim}

\begin{verbatim}
## 
## Attaching package: 'car'
\end{verbatim}

\begin{verbatim}
## The following object is masked from 'package:dplyr':
## 
##     recode
\end{verbatim}

\begin{verbatim}
## The following object is masked from 'package:purrr':
## 
##     some
\end{verbatim}

\begin{Shaded}
\begin{Highlighting}[]
\KeywordTok{EnsurePackage}\NormalTok{(}\StringTok{"MCMCpack"}\NormalTok{) }\CommentTok{# Download MCMCpack package}
\end{Highlighting}
\end{Shaded}

\begin{verbatim}
## Loading required package: MCMCpack
\end{verbatim}

\begin{verbatim}
## Loading required package: coda
\end{verbatim}

\begin{verbatim}
## Loading required package: MASS
\end{verbatim}

\begin{verbatim}
## 
## Attaching package: 'MASS'
\end{verbatim}

\begin{verbatim}
## The following object is masked from 'package:dplyr':
## 
##     select
\end{verbatim}

\begin{verbatim}
## ##
## ## Markov Chain Monte Carlo Package (MCMCpack)
\end{verbatim}

\begin{verbatim}
## ## Copyright (C) 2003-2020 Andrew D. Martin, Kevin M. Quinn, and Jong Hee Park
\end{verbatim}

\begin{verbatim}
## ##
## ## Support provided by the U.S. National Science Foundation
\end{verbatim}

\begin{verbatim}
## ## (Grants SES-0350646 and SES-0350613)
## ##
\end{verbatim}

\begin{Shaded}
\begin{Highlighting}[]
\KeywordTok{EnsurePackage}\NormalTok{(}\StringTok{"dlookr"}\NormalTok{) }\CommentTok{# outlier analysis}
\end{Highlighting}
\end{Shaded}

\begin{verbatim}
## Loading required package: dlookr
\end{verbatim}

\begin{verbatim}
## Loading required package: mice
\end{verbatim}

\begin{verbatim}
## 
## Attaching package: 'mice'
\end{verbatim}

\begin{verbatim}
## The following object is masked from 'package:stats':
## 
##     filter
\end{verbatim}

\begin{verbatim}
## The following objects are masked from 'package:base':
## 
##     cbind, rbind
\end{verbatim}

\begin{verbatim}
## Registered S3 method overwritten by 'quantmod':
##   method            from
##   as.zoo.data.frame zoo
\end{verbatim}

\begin{verbatim}
## 
## Attaching package: 'dlookr'
\end{verbatim}

\begin{verbatim}
## The following object is masked from 'package:base':
## 
##     transform
\end{verbatim}

\begin{Shaded}
\begin{Highlighting}[]
\KeywordTok{EnsurePackage}\NormalTok{(}\StringTok{"mice"}\NormalTok{) }\CommentTok{# missing data }
\KeywordTok{EnsurePackage}\NormalTok{(}\StringTok{"visdat"}\NormalTok{) }\CommentTok{# missing data }
\end{Highlighting}
\end{Shaded}

\begin{verbatim}
## Loading required package: visdat
\end{verbatim}

\begin{Shaded}
\begin{Highlighting}[]
\KeywordTok{cat}\NormalTok{(}\StringTok{"All Packages are available"}\NormalTok{)}
\end{Highlighting}
\end{Shaded}

\begin{verbatim}
## All Packages are available
\end{verbatim}

\begin{Shaded}
\begin{Highlighting}[]
\CommentTok{#import Chile dataset}
\KeywordTok{data}\NormalTok{(Chile)}

\CommentTok{#structure of Chile dataset}
\KeywordTok{str}\NormalTok{(Chile)}
\end{Highlighting}
\end{Shaded}

\begin{verbatim}
## 'data.frame':    2700 obs. of  8 variables:
##  $ region    : Factor w/ 5 levels "C","M","N","S",..: 3 3 3 3 3 3 3 3 3 3 ...
##  $ population: int  175000 175000 175000 175000 175000 175000 175000 175000 175000 175000 ...
##  $ sex       : Factor w/ 2 levels "F","M": 2 2 1 1 1 1 2 1 1 2 ...
##  $ age       : int  65 29 38 49 23 28 26 24 41 41 ...
##  $ education : Factor w/ 3 levels "P","PS","S": 1 2 1 1 3 1 2 3 1 1 ...
##  $ income    : int  35000 7500 15000 35000 35000 7500 35000 15000 15000 15000 ...
##  $ statusquo : num  1.01 -1.3 1.23 -1.03 -1.1 ...
##  $ vote      : Factor w/ 4 levels "A","N","U","Y": 4 2 4 2 2 2 2 2 3 2 ...
\end{verbatim}

\begin{Shaded}
\begin{Highlighting}[]
\CommentTok{# Summary of Chile dataset}
\KeywordTok{summary}\NormalTok{(Chile)}
\end{Highlighting}
\end{Shaded}

\begin{verbatim}
##  region     population     sex           age        education  
##  C :600   Min.   :  3750   F:1379   Min.   :18.00   P   :1107  
##  M :100   1st Qu.: 25000   M:1321   1st Qu.:26.00   PS  : 462  
##  N :322   Median :175000            Median :36.00   S   :1120  
##  S :718   Mean   :152222            Mean   :38.55   NA's:  11  
##  SA:960   3rd Qu.:250000            3rd Qu.:49.00              
##           Max.   :250000            Max.   :70.00              
##                                     NA's   :1                  
##      income         statusquo          vote    
##  Min.   :  2500   Min.   :-1.80301   A   :187  
##  1st Qu.:  7500   1st Qu.:-1.00223   N   :889  
##  Median : 15000   Median :-0.04558   U   :588  
##  Mean   : 33876   Mean   : 0.00000   Y   :868  
##  3rd Qu.: 35000   3rd Qu.: 0.96857   NA's:168  
##  Max.   :200000   Max.   : 2.04859             
##  NA's   :98       NA's   :17
\end{verbatim}

\begin{Shaded}
\begin{Highlighting}[]
\CommentTok{#outlier analysis}
\KeywordTok{diagnose_outlier}\NormalTok{(Chile)}
\end{Highlighting}
\end{Shaded}

\begin{verbatim}
##    variables outliers_cnt outliers_ratio outliers_mean             with_mean
## 1 population            0       0.000000           NaN 152222.22222222221899
## 2        age            0       0.000000           NaN     38.54872174879585
## 3     income          164       6.074074      159756.1  33875.86471944658115
## 4  statusquo            0       0.000000           NaN     -0.00000001118151
##            without_mean
## 1 152222.22222222221899
## 2     38.54872174879585
## 3  25408.12141099261862
## 4     -0.00000001118151
\end{verbatim}

\begin{Shaded}
\begin{Highlighting}[]
\KeywordTok{plot_outlier}\NormalTok{(Chile)}
\end{Highlighting}
\end{Shaded}

\includegraphics{HW9_Rajendiran_files/figure-latex/unnamed-chunk-10-1.pdf}
\includegraphics{HW9_Rajendiran_files/figure-latex/unnamed-chunk-10-2.pdf}
\includegraphics{HW9_Rajendiran_files/figure-latex/unnamed-chunk-10-3.pdf}
\includegraphics{HW9_Rajendiran_files/figure-latex/unnamed-chunk-10-4.pdf}

\begin{Shaded}
\begin{Highlighting}[]
\CommentTok{#data distribution}
\NormalTok{Chile }\OperatorTok\StringTok{ }
\StringTok{  }\KeywordTok{pivot_longer}\NormalTok{(}\DataTypeTok{cols=}\OperatorTok{-}\KeywordTok{c}\NormalTok{(region,sex,vote,education), }\DataTypeTok{names_to=}\StringTok{"variable"}\NormalTok{, }\DataTypeTok{values_to=}\StringTok{"value"}\NormalTok{, }\DataTypeTok{values_drop_na =} \OtherTok{TRUE}\NormalTok{) }\OperatorTok\StringTok{ }
\StringTok{  }\KeywordTok{ggplot}\NormalTok{(}\KeywordTok{aes}\NormalTok{(}\DataTypeTok{x=}\NormalTok{variable, }\DataTypeTok{y=}\NormalTok{value)) }\OperatorTok{+}\StringTok{ }\KeywordTok{geom_violin}\NormalTok{() }\OperatorTok{+}\StringTok{ }\KeywordTok{facet_wrap}\NormalTok{( }\OperatorTok{~}\StringTok{ }\NormalTok{variable, }\DataTypeTok{scales=}\StringTok{"free"}\NormalTok{)}
\end{Highlighting}
\end{Shaded}

\includegraphics{HW9_Rajendiran_files/figure-latex/unnamed-chunk-10-5.pdf}

\begin{Shaded}
\begin{Highlighting}[]
\CommentTok{# missing data }
\KeywordTok{md.pattern}\NormalTok{(Chile, }\DataTypeTok{plot=}\OtherTok{FALSE}\NormalTok{)}
\end{Highlighting}
\end{Shaded}

\begin{verbatim}
##      region population sex age education statusquo income vote    
## 2431      1          1   1   1         1         1      1    1   0
## 150       1          1   1   1         1         1      1    0   1
## 77        1          1   1   1         1         1      0    1   1
## 14        1          1   1   1         1         1      0    0   2
## 8         1          1   1   1         1         0      1    1   1
## 3         1          1   1   1         1         0      1    0   2
## 5         1          1   1   1         1         0      0    1   2
## 9         1          1   1   1         0         1      1    1   1
## 1         1          1   1   1         0         1      0    1   2
## 1         1          1   1   1         0         0      0    0   4
## 1         1          1   1   0         1         1      1    1   1
##           0          0   0   1        11        17     98  168 295
\end{verbatim}

\begin{Shaded}
\begin{Highlighting}[]
\CommentTok{# missing data visualization}
\KeywordTok{vis_miss}\NormalTok{(Chile)}
\end{Highlighting}
\end{Shaded}

\includegraphics{HW9_Rajendiran_files/figure-latex/unnamed-chunk-10-6.pdf}

\begin{Shaded}
\begin{Highlighting}[]
\KeywordTok{options}\NormalTok{(}\DataTypeTok{scipen=}\DecValTok{999}\NormalTok{)  }\CommentTok{# turn-off scientific notation like 1e+48}

\NormalTok{ChileY <-}\StringTok{ }\NormalTok{Chile[Chile}\OperatorTok{$}\NormalTok{vote }\OperatorTok{==}\StringTok{ "Y"}\NormalTok{,] }\CommentTok{# Grab the Yes votes}
\NormalTok{ChileN <-}\StringTok{ }\NormalTok{Chile[Chile}\OperatorTok{$}\NormalTok{vote }\OperatorTok{==}\StringTok{ "N"}\NormalTok{,] }\CommentTok{# Grab the No votes}
\NormalTok{ChileYN <-}\StringTok{ }\KeywordTok{rbind}\NormalTok{(ChileY,ChileN) }\CommentTok{# Make a new dataset with those}
\NormalTok{ChileYN <-}\StringTok{ }\NormalTok{ChileYN[}\KeywordTok{complete.cases}\NormalTok{(ChileYN),] }\CommentTok{# Get rid of missing}
\NormalTok{ChileYN}\OperatorTok{$}\NormalTok{vote <-}\StringTok{ }\KeywordTok{factor}\NormalTok{(ChileYN}\OperatorTok{$}\NormalTok{vote,}\DataTypeTok{levels=}\KeywordTok{c}\NormalTok{(}\StringTok{"N"}\NormalTok{,}\StringTok{"Y"}\NormalTok{)) }\CommentTok{# Fix the factor}

\CommentTok{# Corellation between Agriculture and Fertility}
\KeywordTok{plot}\NormalTok{(ChileYN}\OperatorTok{$}\NormalTok{age, ChileYN}\OperatorTok{$}\NormalTok{statusquo}
\NormalTok{     ,}\DataTypeTok{main=}\StringTok{"Corellation between Age  and statusquo"}
\NormalTok{     ,}\DataTypeTok{col=}\StringTok{"red"}\NormalTok{)}
\end{Highlighting}
\end{Shaded}

\includegraphics{HW9_Rajendiran_files/figure-latex/unnamed-chunk-11-1.pdf}

\begin{Shaded}
\begin{Highlighting}[]
\KeywordTok{str}\NormalTok{(ChileYN)}
\end{Highlighting}
\end{Shaded}

\begin{verbatim}
## 'data.frame':    1703 obs. of  8 variables:
##  $ region    : Factor w/ 5 levels "C","M","N","S",..: 3 3 3 3 3 3 3 3 3 3 ...
##  $ population: int  175000 175000 175000 175000 175000 175000 175000 175000 175000 175000 ...
##  $ sex       : Factor w/ 2 levels "F","M": 2 1 2 1 2 1 1 2 1 2 ...
##  $ age       : int  65 38 64 46 67 38 55 18 24 58 ...
##  $ education : Factor w/ 3 levels "P","PS","S": 1 1 1 3 1 3 2 3 2 1 ...
##  $ income    : int  35000 15000 15000 75000 75000 35000 35000 75000 35000 35000 ...
##  $ statusquo : num  1.01 1.23 1.37 1.51 1.32 ...
##  $ vote      : Factor w/ 2 levels "N","Y": 2 2 2 2 2 2 2 2 2 2 ...
\end{verbatim}

\begin{Shaded}
\begin{Highlighting}[]
\CommentTok{# remove income column}
\NormalTok{ChileYN <-}\StringTok{ }\NormalTok{ChileYN[,}\OperatorTok{!}\KeywordTok{grepl}\NormalTok{(}\StringTok{"income"}\NormalTok{,}\KeywordTok{colnames}\NormalTok{(ChileYN))]}
\NormalTok{ChileYN}\OperatorTok{$}\NormalTok{vote <-}\StringTok{ }\KeywordTok{as.numeric}\NormalTok{(ChileYN}\OperatorTok{$}\NormalTok{vote) }\OperatorTok{-}\StringTok{ }\DecValTok{1} \CommentTok{# Adjust the outcome}
\CommentTok{#table(ChileYN$vote)}

\KeywordTok{str}\NormalTok{(ChileYN)}
\end{Highlighting}
\end{Shaded}

\begin{verbatim}
## 'data.frame':    1703 obs. of  7 variables:
##  $ region    : Factor w/ 5 levels "C","M","N","S",..: 3 3 3 3 3 3 3 3 3 3 ...
##  $ population: int  175000 175000 175000 175000 175000 175000 175000 175000 175000 175000 ...
##  $ sex       : Factor w/ 2 levels "F","M": 2 1 2 1 2 1 1 2 1 2 ...
##  $ age       : int  65 38 64 46 67 38 55 18 24 58 ...
##  $ education : Factor w/ 3 levels "P","PS","S": 1 1 1 3 1 3 2 3 2 1 ...
##  $ statusquo : num  1.01 1.23 1.37 1.51 1.32 ...
##  $ vote      : num  1 1 1 1 1 1 1 1 1 1 ...
\end{verbatim}

\begin{Shaded}
\begin{Highlighting}[]
\KeywordTok{summary}\NormalTok{(ChileYN}\OperatorTok{$}\NormalTok{statusquo)}
\end{Highlighting}
\end{Shaded}

\begin{verbatim}
##     Min.  1st Qu.   Median     Mean  3rd Qu.     Max. 
## -1.72594 -1.09671 -0.18511 -0.00467  1.16602  1.71355
\end{verbatim}

\begin{Shaded}
\begin{Highlighting}[]
\KeywordTok{summary}\NormalTok{(ChileYN)}
\end{Highlighting}
\end{Shaded}

\begin{verbatim}
##  region     population     sex          age        education   statusquo       
##  C :374   Min.   :  3750   F:814   Min.   :18.00   P :671    Min.   :-1.72594  
##  M : 54   1st Qu.: 25000   M:889   1st Qu.:25.00   PS:343    1st Qu.:-1.09671  
##  N :230   Median :175000           Median :36.00   S :689    Median :-0.18511  
##  S :476   Mean   :150716           Mean   :38.06             Mean   :-0.00467  
##  SA:569   3rd Qu.:250000           3rd Qu.:49.00             3rd Qu.: 1.16602  
##           Max.   :250000           Max.   :70.00             Max.   : 1.71355  
##       vote       
##  Min.   :0.0000  
##  1st Qu.:0.0000  
##  Median :0.0000  
##  Mean   :0.4909  
##  3rd Qu.:1.0000  
##  Max.   :1.0000
\end{verbatim}

\begin{Shaded}
\begin{Highlighting}[]
\CommentTok{# missing data }
\KeywordTok{md.pattern}\NormalTok{(ChileYN, }\DataTypeTok{plot=}\OtherTok{FALSE}\NormalTok{)}
\end{Highlighting}
\end{Shaded}

\begin{verbatim}
##  /\     /\
## {  `---'  }
## {  O   O  }
## ==>  V <==  No need for mice. This data set is completely observed.
##  \  \|/  /
##   `-----'
\end{verbatim}

\begin{verbatim}
##      region population sex age education statusquo vote  
## 1703      1          1   1   1         1         1    1 0
##           0          0   0   0         0         0    0 0
\end{verbatim}

\begin{Shaded}
\begin{Highlighting}[]
\CommentTok{# Crete glm model}
\NormalTok{chile.glm <-}\StringTok{ }\KeywordTok{glm}\NormalTok{(vote }\OperatorTok{~}\StringTok{ }\NormalTok{age }\OperatorTok{+}\StringTok{ }\NormalTok{statusquo,}\DataTypeTok{data=}\NormalTok{ChileYN,}\DataTypeTok{family =} \KeywordTok{binomial}\NormalTok{(}\DataTypeTok{link =} \StringTok{"logit"}\NormalTok{)) }
\KeywordTok{summary}\NormalTok{(chile.glm)}
\end{Highlighting}
\end{Shaded}

\begin{verbatim}
## 
## Call:
## glm(formula = vote ~ age + statusquo, family = binomial(link = "logit"), 
##     data = ChileYN)
## 
## Deviance Residuals: 
##     Min       1Q   Median       3Q      Max  
## -3.2095  -0.2830  -0.1840   0.1889   2.8789  
## 
## Coefficients:
##              Estimate Std. Error z value            Pr(>|z|)    
## (Intercept) -0.193759   0.270708  -0.716              0.4741    
## age          0.011322   0.006826   1.659              0.0972 .  
## statusquo    3.174487   0.143921  22.057 <0.0000000000000002 ***
## ---
## Signif. codes:  0 '***' 0.001 '**' 0.01 '*' 0.05 '.' 0.1 ' ' 1
## 
## (Dispersion parameter for binomial family taken to be 1)
## 
##     Null deviance: 2360.29  on 1702  degrees of freedom
## Residual deviance:  734.52  on 1700  degrees of freedom
## AIC: 740.52
## 
## Number of Fisher Scoring iterations: 6
\end{verbatim}

\begin{Shaded}
\begin{Highlighting}[]
\CommentTok{#histogram on residuals chile.glm}
\KeywordTok{hist}\NormalTok{(}\KeywordTok{residuals}\NormalTok{(chile.glm))}
\end{Highlighting}
\end{Shaded}

\includegraphics{HW9_Rajendiran_files/figure-latex/unnamed-chunk-11-2.pdf}

\begin{Shaded}
\begin{Highlighting}[]
\CommentTok{#Chi-Square analysis on logistic regression}
\KeywordTok{anova}\NormalTok{(chile.glm, }\DataTypeTok{test=}\StringTok{"Chisq"}\NormalTok{)}
\end{Highlighting}
\end{Shaded}

\begin{verbatim}
## Analysis of Deviance Table
## 
## Model: binomial, link: logit
## 
## Response: vote
## 
## Terms added sequentially (first to last)
## 
## 
##           Df Deviance Resid. Df Resid. Dev              Pr(>Chi)    
## NULL                       1702    2360.29                          
## age        1     34.2      1701    2326.09        0.000000004964 ***
## statusquo  1   1591.6      1700     734.52 < 0.00000000000000022 ***
## ---
## Signif. codes:  0 '***' 0.001 '**' 0.01 '*' 0.05 '.' 0.1 ' ' 1
\end{verbatim}

\begin{Shaded}
\begin{Highlighting}[]
\CommentTok{# Convert the log odds for the coefficient on the predictor into regular odds}
\KeywordTok{exp}\NormalTok{(}\KeywordTok{coef}\NormalTok{(chile.glm))}
\end{Highlighting}
\end{Shaded}

\begin{verbatim}
## (Intercept)         age   statusquo 
##   0.8238564   1.0113863  23.9145451
\end{verbatim}

\begin{Shaded}
\begin{Highlighting}[]
\CommentTok{#confusion matrix}
\KeywordTok{table}\NormalTok{(}\KeywordTok{round}\NormalTok{(}\KeywordTok{predict}\NormalTok{(chile.glm,}\DataTypeTok{type=}\StringTok{"response"}\NormalTok{)),ChileYN}\OperatorTok{$}\NormalTok{vote)}
\end{Highlighting}
\end{Shaded}

\begin{verbatim}
##    
##       0   1
##   0 810  74
##   1  57 762
\end{verbatim}

\begin{Shaded}
\begin{Highlighting}[]
\KeywordTok{set.seed}\NormalTok{(}\DecValTok{271}\NormalTok{) }\CommentTok{# Control randomization}
\CommentTok{#bayesian estimation of logistic regression}
\NormalTok{bayesLogitOut <-}\StringTok{ }\KeywordTok{MCMClogit}\NormalTok{(}\DataTypeTok{formula =}\NormalTok{ vote }\OperatorTok{~}\StringTok{ }\NormalTok{age }\OperatorTok{+}\StringTok{ }\NormalTok{statusquo, }\DataTypeTok{data =}\NormalTok{ ChileYN)}
\KeywordTok{summary}\NormalTok{(bayesLogitOut) }\CommentTok{# Summarize the results}
\end{Highlighting}
\end{Shaded}

\begin{verbatim}
## 
## Iterations = 1001:11000
## Thinning interval = 1 
## Number of chains = 1 
## Sample size per chain = 10000 
## 
## 1. Empirical mean and standard deviation for each variable,
##    plus standard error of the mean:
## 
##                 Mean       SD   Naive SE Time-series SE
## (Intercept) -0.18272 0.272640 0.00272640       0.008938
## age          0.01123 0.006817 0.00006817       0.000223
## statusquo    3.19061 0.145853 0.00145853       0.004993
## 
## 2. Quantiles for each variable:
## 
##                  2.5%       25%      50%        75%   97.5%
## (Intercept) -0.742761 -0.365241 -0.17552 -0.0003872 0.34439
## age         -0.002005  0.006733  0.01121  0.0157683 0.02499
## statusquo    2.914442  3.087259  3.18546  3.2847388 3.48698
\end{verbatim}

\begin{Shaded}
\begin{Highlighting}[]
\KeywordTok{plot}\NormalTok{(bayesLogitOut)}
\end{Highlighting}
\end{Shaded}

\includegraphics{HW9_Rajendiran_files/figure-latex/unnamed-chunk-11-3.pdf}

\begin{Shaded}
\begin{Highlighting}[]
\KeywordTok{EnsurePackage}\NormalTok{(}\StringTok{"MASS"}\NormalTok{) }\CommentTok{# AIC}

\NormalTok{stepOut <-}\StringTok{ }\KeywordTok{stepAIC}\NormalTok{(chile.glm)}
\end{Highlighting}
\end{Shaded}

\begin{verbatim}
## Start:  AIC=740.52
## vote ~ age + statusquo
## 
##             Df Deviance     AIC
## <none>           734.52  740.52
## - age        1   737.27  741.27
## - statusquo  1  2326.09 2330.09
\end{verbatim}

\begin{Shaded}
\begin{Highlighting}[]
\NormalTok{stepOut}\OperatorTok{$}\NormalTok{anova}
\end{Highlighting}
\end{Shaded}

\begin{verbatim}
## Stepwise Model Path 
## Analysis of Deviance Table
## 
## Initial Model:
## vote ~ age + statusquo
## 
## Final Model:
## vote ~ age + statusquo
## 
## 
##   Step Df Deviance Resid. Df Resid. Dev      AIC
## 1                       1700   734.5207 740.5207
\end{verbatim}

\begin{Shaded}
\begin{Highlighting}[]
\CommentTok{# stepOutOLD <- stepAIC(chout)}
\CommentTok{# stepOutOLD$anova}
\end{Highlighting}
\end{Shaded}

\hypertarget{chapter-10-exercise-7}{%
\section{Chapter 10, Exercise 7}\label{chapter-10-exercise-7}}

\emph{Bonus R code question: Develop your own custom function that will
take the posterior distribution of a coefficient from the output object
from an MCMClogit() analysis and automatically create a histogram of the
posterior distributions of the coefficient in terms of regular odds
(instead of log‐odds). Make sure to mark vertical lines on the histogram
indicating the boundaries of the 95\% HDI. (1 pt) Run the function on
your regression results. (1 pt)}

\begin{itemize}
\tightlist
\item
  Custom Function:

  \begin{itemize}
  \tightlist
  \item
    histquoLogOdds takes one of the predictor variable output (log odds)
    as input
  \item
    converts the log-odds into regular odds with exponential function as
    exp(quoLogOdds)
  \item
    hist(quoLogOdds, main=prdictor,col=``grey'') - procedure creates
    histogram with HDI lines
  \item
    HDI lines

    \begin{itemize}
    \tightlist
    \item
      2.5\% or HDI lower bound \textbar{}
      abline(v=quantile(quoLogOdds,c(0.025)),col=``orange'')
    \item
      97.5\% or HDI upper bound \textbar{}
      abline(v=quantile(quoLogOdds,c(0.975)),col=``blue'')
    \item
      median or 50\% percent quantile \textbar{}
      abline(v=quantile(quoLogOdds,c(0.50)),col=``green'')
    \end{itemize}
  \end{itemize}
\item
  Output of the function produces histograms

  \begin{itemize}
  \tightlist
  \item
    histquoLogOdds(``statusquo'') - histogram on statusquo regular odds

    \begin{itemize}
    \tightlist
    \item
      Status histograms has values between\textasciitilde{}15 to
      \textasciitilde{}34 as lower and upper bounds
    \end{itemize}
  \item
    histquoLogOdds(``age'') - histogram on age regular odds

    \begin{itemize}
    \tightlist
    \item
      Age histograms has more consistent values between just below 1 and
      over 1.
    \end{itemize}
  \end{itemize}
\end{itemize}

\hypertarget{please-find-more-details-from-the-r-code-below-3}{%
\section{Please find more details from the R code
below,}\label{please-find-more-details-from-the-r-code-below-3}}

\begin{Shaded}
\begin{Highlighting}[]
\CommentTok{# bayesLogitOut hist }

\CommentTok{# Custom function to output histograms with HDI vertical lines }
\CommentTok{# prdictor bayesLogitOut - output variable coeffients converted to regular odds (instead of log-odds)}
\NormalTok{histquoLogOdds <-}\StringTok{ }\ControlFlowTok{function}\NormalTok{(prdictor)}
\NormalTok{  \{}
\NormalTok{  quoLogOdds <-}\StringTok{ }\KeywordTok{as.matrix}\NormalTok{(bayesLogitOut[,prdictor])}
\NormalTok{  quoLogOdds <-}\StringTok{ }\KeywordTok{exp}\NormalTok{(quoLogOdds)  }\CommentTok{# regular odds (instead of log‐odds)}
  \KeywordTok{hist}\NormalTok{(quoLogOdds, }\DataTypeTok{main=}\NormalTok{prdictor,}\DataTypeTok{col=}\StringTok{"grey"}\NormalTok{) }\CommentTok{# hist}
  \KeywordTok{abline}\NormalTok{(}\DataTypeTok{v=}\KeywordTok{quantile}\NormalTok{(quoLogOdds,}\KeywordTok{c}\NormalTok{(}\FloatTok{0.025}\NormalTok{)),}\DataTypeTok{col=}\StringTok{"orange"}\NormalTok{) }
  \KeywordTok{abline}\NormalTok{(}\DataTypeTok{v=}\KeywordTok{quantile}\NormalTok{(quoLogOdds,}\KeywordTok{c}\NormalTok{(}\FloatTok{0.975}\NormalTok{)),}\DataTypeTok{col=}\StringTok{"blue"}\NormalTok{) }
  \KeywordTok{abline}\NormalTok{(}\DataTypeTok{v=}\KeywordTok{quantile}\NormalTok{(quoLogOdds,}\KeywordTok{c}\NormalTok{(}\FloatTok{0.50}\NormalTok{)),}\DataTypeTok{col=}\StringTok{"green"}\NormalTok{) }
\NormalTok{  \}}

\CommentTok{#Plot histograms}
\KeywordTok{histquoLogOdds}\NormalTok{(}\StringTok{"statusquo"}\NormalTok{) }
\end{Highlighting}
\end{Shaded}

\includegraphics{HW9_Rajendiran_files/figure-latex/unnamed-chunk-14-1.pdf}

\begin{Shaded}
\begin{Highlighting}[]
\KeywordTok{histquoLogOdds}\NormalTok{(}\StringTok{"age"}\NormalTok{) }
\end{Highlighting}
\end{Shaded}

\includegraphics{HW9_Rajendiran_files/figure-latex/unnamed-chunk-14-2.pdf}

\end{document}
